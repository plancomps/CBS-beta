% 


\begin{center}
\rule{3in}{0.4pt}
\end{center}

\subsection{Threads}\hypertarget{threads}{}\label{threads}

\begin{align*}
  [ \
  \textsf{Multithreading
          } \ & \textsf{} \\
  \textsf{Synchronising
          } \ & \textsf{} \\
  \textsf{Locks
          } \ & \textsf{} \\
  \textsf{Notifications
          } \ & \textsf{}
  \ ]
\end{align*}
The funcons for threads are tentative. They have not yet been rigorously
unit-tested, nor used significantly in language definitions.

The \href{Multithreading}{multithreading} funcons involve multiple mutable entities, and are
generally specified by inference rules with premises involving the values of
those entities before and after a transition.

The \href{Synchronising}{synchronising} funcons only involve the $\SHADE{\NAMEHYPER{../../../Funcons-beta/Computations/Normal}{Storing}{store}}$ entity, and wrap
compound funcon terms in $\SHADE{\NAME{thread-atomic}
           (  \_ )}$ to inhibit preemption.

Some of the unit tests are based on examples in an \href{https://www.ibm.com/support/knowledgecenter/ssw_aix_72/com.ibm.aix.genprogc/chapter12.htm}{IBM threads} guide.

% 


