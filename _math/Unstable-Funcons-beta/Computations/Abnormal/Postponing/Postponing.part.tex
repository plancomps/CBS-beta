% 


\begin{center}
\rule{3in}{0.4pt}
\end{center}

\subsubsection{Postponing}\hypertarget{postponing}{}\label{postponing}

\begin{align*}
  [ \
  \KEY{Entity} \quad & \NAMEREF{postponing} \\
  \KEY{Funcon} \quad & \NAMEREF{postpone} \\
  \KEY{Funcon} \quad & \NAMEREF{postpone-after-effect} \\
  \KEY{Funcon} \quad & \NAMEREF{after-effect}
  \ ]
\end{align*}
A funcon term can extend itself (e.g., with code to release the resources
allocated to it) using general funcons for postponed execution. When a step
from $\SHADE{\VAR{X}}$ to $\SHADE{\VAR{X}'}$ executes $\SHADE{\NAMEREF{postpone}
           (  \VAR{Y} )}$ (which computes $\SHADE{\NAMEHYPER{../../../../Funcons-beta/Values/Primitive}{Null}{null}}$),
the corresponding step of  $\SHADE{\NAMEREF{postpone-after-effect}
           (  \VAR{X} )}$ gives
$\SHADE{\NAMEREF{postpone-after-effect}
           (  \NAMEREF{after-effect}
                   (  \VAR{X}', 
                          \VAR{Y} ) )}$, so that normal termination
of $\SHADE{\VAR{X}'}$ is followed by the effect of $\SHADE{\VAR{Y}}$.

The control entity $\SHADE{\NAMEREF{postponing}
           (  \VAR{A} )}$ signals that the execution of the body
of the abstraction $\SHADE{\VAR{A}}$ is postponed:

\begin{align*}
  \KEY{Entity} \quad
  & \_ \xrightarrow{\NAMEDECL{postponing}(\_ : (  \NAMEHYPER{../../../../Funcons-beta/Values/Abstraction}{Generic}{abstractions}
                                                                      (   \TO \NAMEHYPER{../../../../Funcons-beta/Values/Primitive}{Null}{null-type} ) )\QUERY)} \_
\end{align*}
The funcon $\SHADE{\NAMEREF{postpone}
           (  \VAR{X} )}$ forms a closure from $\SHADE{\VAR{X}}$ and signals that its
execution is postponed:

\begin{align*}
  \KEY{Funcon} \quad
  & \NAMEDECL{postpone}(
                       \_ :  \TO \NAMEHYPER{../../../../Funcons-beta/Values}{Value-Types}{values}) 
    :  \TO \NAMEHYPER{../../../../Funcons-beta/Values/Primitive}{Null}{null-type} 
\\
  \KEY{Rule} \quad
    & \RULE{
      & \NAMEHYPER{../../../../Funcons-beta/Computations/Normal}{Giving}{given-value} (  \VAR{V} ) \vdash \NAMEHYPER{../../../../Funcons-beta/Values/Abstraction}{Generic}{closure} \ 
                      \NAMEHYPER{../../../../Funcons-beta/Computations/Normal}{Giving}{give}
                        (  \VAR{V}, 
                               \VAR{X} ) \xrightarrow{\NAMEREF{postponing}(   \  )}_{} 
          \VAR{A}
      }{
      & \NAMEHYPER{../../../../Funcons-beta/Computations/Normal}{Giving}{given-value} (  \VAR{V} ) \vdash \NAMEREF{postpone}
                      (  \VAR{X} ) \xrightarrow{\NAMEREF{postponing}(  \VAR{A} )}_{} 
          \NAMEHYPER{../../../../Funcons-beta/Values/Primitive}{Null}{null-value}
      }
\\
  \KEY{Rule} \quad
    & \RULE{
      & \NAMEHYPER{../../../../Funcons-beta/Computations/Normal}{Giving}{given-value} (   \  ) \vdash \NAMEHYPER{../../../../Funcons-beta/Values/Abstraction}{Generic}{closure} \ 
                      \NAMEHYPER{../../../../Funcons-beta/Computations/Normal}{Giving}{no-given} \ 
                        \VAR{X} \xrightarrow{\NAMEREF{postponing}(   \  )}_{} 
          \VAR{A}
      }{
      & \NAMEHYPER{../../../../Funcons-beta/Computations/Normal}{Giving}{given-value} (   \  ) \vdash \NAMEREF{postpone}
                      (  \VAR{X} ) \xrightarrow{\NAMEREF{postponing}(  \VAR{A} )}_{} 
          \NAMEHYPER{../../../../Funcons-beta/Values/Primitive}{Null}{null-value}
      }
\end{align*}
The funcon $\SHADE{\NAMEREF{postpone-after-effect}
           (  \VAR{X} )}$ handles each signal $\SHADE{\NAMEREF{postponing}
           (  \VAR{A} )}$
by adding it as an after-effect of $\SHADE{\VAR{X}}$:

\begin{align*}
  \KEY{Funcon} \quad
  & \NAMEDECL{postpone-after-effect}(
                       \_ :  \TO \VAR{T}) 
    :  \TO \VAR{T} 
\\
  \KEY{Rule} \quad
    & \RULE{
      &  \VAR{X} \xrightarrow{\NAMEREF{postponing}(   \  )}_{} 
          \VAR{X}'
      }{
      &  \NAMEREF{postpone-after-effect}
                      (  \VAR{X} ) \xrightarrow{\NAMEREF{postponing}(   \  )}_{} \\&\quad
          \NAMEREF{postpone-after-effect}
            (  \VAR{X}' )
      }
\\
  \KEY{Rule} \quad
    & \RULE{
      &  \VAR{X} \xrightarrow{\NAMEREF{postponing}(  \VAR{A} )}_{} 
          \VAR{X}'\\&
        \VAR{A} \leadsto 
          \NAMEHYPER{../../../../Funcons-beta/Values/Abstraction}{Generic}{abstraction} \ 
            \VAR{Y}
      }{
      &  \NAMEREF{postpone-after-effect}
                      (  \VAR{X} ) \xrightarrow{\NAMEREF{postponing}(   \  )}_{} \\&\quad
          \NAMEREF{postpone-after-effect}
            (  \NAMEREF{after-effect}
                    (  \VAR{X}', 
                           \VAR{Y} ) )
      }
\\
  \KEY{Rule} \quad
    & \NAMEREF{postpone-after-effect}
        (  \VAR{V} : \NAMEHYPER{../../../../Funcons-beta/Values}{Value-Types}{values} ) \leadsto 
        \VAR{V}
\end{align*}
The funcon $\SHADE{\NAMEREF{after-effect}
           (  \VAR{X}, 
                  \VAR{Y} )}$ first executes $\SHADE{\VAR{X}}$. If $\SHADE{\VAR{X}}$ computes a value $\SHADE{\VAR{V}}$,
it then executes $\SHADE{\VAR{Y}}$, and computes $\SHADE{\VAR{V}}$:

\begin{align*}
  \KEY{Funcon} \quad
  & \NAMEDECL{after-effect}(
                       \VAR{X} :  \TO \VAR{T}, \VAR{Y} :  \TO \NAMEHYPER{../../../../Funcons-beta/Values/Primitive}{Null}{null-type}) 
    :  \TO \VAR{T} \\&\quad
    \leadsto \NAMEHYPER{../../../../Funcons-beta/Computations/Normal}{Giving}{give}
               (  \VAR{X}, 
                      \NAMEHYPER{../../../../Funcons-beta/Computations/Normal}{Flowing}{sequential}
                       (  \VAR{Y}, 
                              \NAMEHYPER{../../../../Funcons-beta/Computations/Normal}{Giving}{given} ) )
\end{align*}
% 


