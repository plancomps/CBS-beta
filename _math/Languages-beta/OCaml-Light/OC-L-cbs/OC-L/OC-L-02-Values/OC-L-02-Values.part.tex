% 



    OUTLINE
  \tableofcontents
\begin{center}
\rule{3in}{0.4pt}
\end{center}

\begin{displaymath}
\KEY{Language} \quad \STRING{OCaml Light}
\end{displaymath}

\section{$\SECT{2}$ Values}\hypertarget{SectionNumber:2}{}\label{SectionNumber:2}

The comments below are excerpts from section 7.2 of \href{https://ocaml.org/releases/4.06/htmlman/values.html}{The OCaml System,
  release 4.06}.

\begin{align*}
  \KEY{Type} \quad 
  & \NAMEDECL{implemented-values}  \\&\quad
    \leadsto \NAMEHYPER{../../../../../Funcons-beta/Values/Primitive}{Null}{null-type} \\&\quad\quad\quad \mid \NAMEHYPER{../../../../../Funcons-beta/Values/Primitive}{Booleans}{booleans} \\&\quad\quad\quad \mid \NAMEREF{implemented-integers} \\&\quad\quad\quad \mid \NAMEREF{implemented-floats} \\&\quad\quad\quad \mid \NAMEREF{implemented-characters} \\&\quad\quad\quad \mid \NAMEREF{implemented-strings} \\&\quad\quad\quad \mid \NAMEREF{implemented-tuples} \\&\quad\quad\quad \mid \NAMEREF{implemented-lists} \\&\quad\quad\quad \mid \NAMEREF{implemented-records} \\&\quad\quad\quad \mid \NAMEREF{implemented-references} \\&\quad\quad\quad \mid \NAMEREF{implemented-vectors} \\&\quad\quad\quad \mid \NAMEREF{implemented-variants} \\&\quad\quad\quad \mid \NAMEREF{implemented-functions}
\end{align*}
\subsection{Base values}\hypertarget{base-values}{}\label{base-values}

\subsubsection{Integer numbers}\hypertarget{integer-numbers}{}\label{integer-numbers}

Integer values are integer numbers from -2\^{}\{30\} to 2\^{}\{30\}-1, 
  that is -1073741824 to 1073741823. The implementation may support a wider
  range of integer values (\ldots{}).

\begin{align*}
  \KEY{Type} \quad 
  & \NAMEDECL{implemented-integers}  \\&\quad
    \leadsto \NAMEHYPER{../../../../../Funcons-beta/Values/Primitive}{Integers}{integers}
\end{align*}
\begin{align*}
  \KEY{Funcon} \quad
  & \NAMEDECL{implemented-integer}(
                       \VAR{I} : \NAMEHYPER{../../../../../Funcons-beta/Values/Primitive}{Integers}{integers}) 
    :  \TO \NAMEREF{implemented-integers} \\&\quad
    \leadsto \VAR{I}
\\
  \KEY{Assert} \quad
  & \NAMEHYPER{../../../../../Funcons-beta/Values}{Value-Types}{is-equal}
      ( \\&\quad \NAMEHYPER{../../../../../Funcons-beta/Values/Primitive}{Null}{null}, \\&\quad
             \NAMEREF{implemented-integer}
              (  \VAR{N} : \NAMEHYPER{../../../../../Funcons-beta/Values/Primitive}{Integers}{bounded-integers}
                                (  -1073741824, 
                                       1073741823 ) ) ) \\&\quad
    == \NAMEHYPER{../../../../../Funcons-beta/Values/Primitive}{Booleans}{false}
\end{align*}
\begin{align*}
  \KEY{Funcon} \quad
  & \NAMEDECL{implemented-integers-width} 
    :  \TO \NAMEHYPER{../../../../../Funcons-beta/Values/Primitive}{Integers}{natural-numbers} \\&\quad
    \leadsto 31
\end{align*}
\begin{align*}
  \KEY{Funcon} \quad
  & \NAMEDECL{implemented-integer-literal}(
                       \VAR{IL} : \NAMEHYPER{../../../../../Funcons-beta/Values/Composite}{Strings}{strings}) 
    :  \TO \NAMEREF{implemented-integers} \\&\quad
    \leadsto \NAMEREF{implemented-integer} \ 
               \NAMEHYPER{../../../../../Funcons-beta/Values/Primitive}{Integers}{decimal-natural}
                 (  \VAR{IL} )
\end{align*}
\begin{align*}
  \KEY{Funcon} \quad
  & \NAMEDECL{implemented-bit-vector}(
                       \VAR{I} : \NAMEREF{implemented-integers}) \\&\quad
    :  \TO \NAMEHYPER{../../../../../Funcons-beta/Values/Composite}{Bits}{bit-vectors}
                     (  \NAMEREF{implemented-integers-width} ) \\&\quad
    \leadsto \NAMEHYPER{../../../../../Funcons-beta/Values/Composite}{Bits}{integer-to-bit-vector}
               (  \VAR{I}, 
                      \NAMEREF{implemented-integers-width} )
\end{align*}
\subsubsection{Floating-point numbers}\hypertarget{floating-point-numbers}{}\label{floating-point-numbers}

Floating-point values are numbers in floating-point representation. 
  The current implementation uses double-precision floating-point numbers
  conforming to the IEEE 754 standard, with 53 bits of mantissa and
  an exponent ranging from -1022 to 1023.

\begin{align*}
  \KEY{Type} \quad 
  & \NAMEDECL{implemented-floats}  
\end{align*}
\begin{align*}
  \KEY{Funcon} \quad
  & \NAMEDECL{implemented-floats-format} 
    :  \TO \NAMEHYPER{../../../../../Funcons-beta/Values/Primitive}{Floats}{float-formats} \\&\quad
    \leadsto \NAMEHYPER{../../../../../Funcons-beta/Values/Primitive}{Floats}{binary64}
\end{align*}
\begin{align*}
  \KEY{Funcon} \quad
  & \NAMEDECL{implemented-float-literal}(
                       \VAR{FL} : \NAMEHYPER{../../../../../Funcons-beta/Values/Composite}{Strings}{strings}) 
    :  \TO \NAMEREF{implemented-floats} 
\end{align*}
\subsubsection{Characters}\hypertarget{characters}{}\label{characters}

Character values are represented as 8-bit integers between 0 and 255.
  Character codes between 0 and 127 are interpreted following the ASCII
  standard. The current implementation interprets character codes between 
  128 and 255 following the ISO 8859-1 standard.

\begin{align*}
  \KEY{Type} \quad 
  & \NAMEDECL{implemented-characters} <: \NAMEHYPER{../../../../../Funcons-beta/Values/Primitive}{Characters}{characters} 
\end{align*}
\begin{align*}
  \KEY{Type} \quad 
  & \NAMEDECL{implemented-character-points}  \\&\quad
    \leadsto \NAMEHYPER{../../../../../Funcons-beta/Values/Primitive}{Integers}{bounded-integers}
               (  0, 
                      255 )
\end{align*}
\begin{align*}
  \KEY{Funcon} \quad
  & \NAMEDECL{implemented-character}(
                       \VAR{C} : \NAMEHYPER{../../../../../Funcons-beta/Values/Primitive}{Characters}{characters}) 
    :  \TO \NAMEREF{implemented-characters}\QUERY \\&\quad
    \leadsto \NAMEHYPER{../../../../../Funcons-beta/Values/Primitive}{Characters}{ascii-character} \ 
               [  \VAR{C} ]
\end{align*}
\subsubsection{Character strings}\hypertarget{character-strings}{}\label{character-strings}

String values are finite sequences of characters. The current implementation
  supports strings containing up to 2\^{}24 - 5 characters (16777211 characters);
  (\ldots{})

\begin{align*}
  \KEY{Type} \quad 
  & \NAMEDECL{implemented-strings} <: \NAMEHYPER{../../../../../Funcons-beta/Values/Composite}{Lists}{lists}
                                     (  \NAMEREF{implemented-characters} ) 
\end{align*}
\begin{align*}
  \KEY{Funcon} \quad
  & \NAMEDECL{implemented-string}(
                       \VAR{L} : \NAMEHYPER{../../../../../Funcons-beta/Values/Composite}{Lists}{lists}
                                 (  \NAMEREF{implemented-characters} )) 
    :  \TO \NAMEREF{implemented-strings}\QUERY \\&\quad
    \leadsto \NAMEHYPER{../../../../../Funcons-beta/Values}{Value-Types}{when-true}
               (  \NAMEHYPER{../../../../../Funcons-beta/Values/Primitive}{Integers}{is-less-or-equal}
                       (  \NAMEHYPER{../../../../../Funcons-beta/Values/Composite}{Sequences}{length} \ 
                               \NAMEHYPER{../../../../../Funcons-beta/Values/Composite}{Lists}{list-elements} \ 
                                 \VAR{L}, 
                              16777211 ), 
                      \VAR{L} )
\end{align*}
\subsection{Tuples}\hypertarget{tuples}{}\label{tuples}

Tuples of values are written (v\_1, \ldots{}, v\_n), standing for the n-tuple of
  values v\_1 to v\_n. The current implementation supports tuples of up to 
  2\^{}22 - 1 elements (4194303 elements).

\begin{align*}
  \KEY{Type} \quad 
  & \NAMEDECL{implemented-tuples} <: \NAMEHYPER{../../../../../Funcons-beta/Values/Composite}{Tuples}{tuples}
                                     (  \NAMEREF{implemented-values}\STAR ) \\&\quad
    \leadsto \NAMEHYPER{../../../../../Funcons-beta/Values/Composite}{Tuples}{tuples}
               (  \NAMEHYPER{../../../../../Funcons-beta/Values}{Value-Types}{values}\STAR )
\end{align*}
\begin{align*}
  \KEY{Funcon} \quad
  & \NAMEDECL{implemented-tuple}(
                       \VAR{T} : \NAMEHYPER{../../../../../Funcons-beta/Values/Composite}{Tuples}{tuples}
                                 (  \NAMEHYPER{../../../../../Funcons-beta/Values}{Value-Types}{values}\STAR )) 
    :  \TO \NAMEREF{implemented-tuples}\QUERY \\&\quad
    \leadsto \NAMEHYPER{../../../../../Funcons-beta/Values}{Value-Types}{when-true}
               (  \NAMEHYPER{../../../../../Funcons-beta/Values/Primitive}{Integers}{is-less-or-equal}
                       (  \NAMEHYPER{../../../../../Funcons-beta/Values/Composite}{Sequences}{length} \ 
                               \NAMEHYPER{../../../../../Funcons-beta/Values/Composite}{Tuples}{tuple-elements} \ 
                                 \VAR{T}, 
                              4194303 ), 
                      \VAR{T} )
\end{align*}
In OCaml Light, the unit value is represented by $\SHADE{\NAMEHYPER{../../../../../Funcons-beta/Values/Composite}{Tuples}{tuple}
           (   \  )}$.

In OCaml Light, lists are written {[}v\_1; \ldots{}; v\_n{]}, and their values are
  represented by list values in CBS.

\begin{align*}
  \KEY{Type} \quad 
  & \NAMEDECL{implemented-lists} <: \NAMEHYPER{../../../../../Funcons-beta/Values/Composite}{Lists}{lists}
                                     (  \NAMEREF{implemented-values} ) \\&\quad
    \leadsto \NAMEHYPER{../../../../../Funcons-beta/Values/Composite}{Lists}{lists}
               (  \NAMEHYPER{../../../../../Funcons-beta/Values}{Value-Types}{values} )
\end{align*}
\begin{align*}
  \KEY{Funcon} \quad
  & \NAMEDECL{implemented-list}(
                       \VAR{L} : \NAMEHYPER{../../../../../Funcons-beta/Values/Composite}{Lists}{lists}
                                 (  \NAMEHYPER{../../../../../Funcons-beta/Values}{Value-Types}{values} )) 
    :  \TO \NAMEREF{implemented-lists}\QUERY \\&\quad
    \leadsto \NAMEHYPER{../../../../../Funcons-beta/Values}{Value-Types}{when-true}
               (  \NAMEHYPER{../../../../../Funcons-beta/Values/Primitive}{Integers}{is-less-or-equal}
                       (  \NAMEHYPER{../../../../../Funcons-beta/Values/Composite}{Sequences}{length} \ 
                               \NAMEHYPER{../../../../../Funcons-beta/Values/Composite}{Lists}{list-elements} \ 
                                 \VAR{L}, 
                              4194303 ), 
                      \VAR{L} )
\end{align*}
\subsection{Records}\hypertarget{records}{}\label{records}

Record values are labeled tuples of values. The record value written 
  \{ field\_1 = v\_1; \ldots{}; field\_n = v\_n \} associates the value v\_i to the
  record field field\_i, for i = 1 \ldots{} n. The current implementation supports
  records with up to 2\^{}22 - 1 fields (4194303 fields).

\begin{align*}
  \KEY{Type} \quad 
  & \NAMEDECL{implemented-records} <: \NAMEHYPER{../../../../../Funcons-beta/Values/Composite}{Records}{records}
                                     (  \NAMEREF{implemented-values} ) \\&\quad
    \leadsto \NAMEHYPER{../../../../../Funcons-beta/Values/Composite}{Records}{records}
               (  \NAMEHYPER{../../../../../Funcons-beta/Values}{Value-Types}{values} )
\end{align*}
\begin{align*}
  \KEY{Funcon} \quad
  & \NAMEDECL{implemented-record}(
                       \VAR{R} : \NAMEHYPER{../../../../../Funcons-beta/Values/Composite}{Records}{records}
                                 (  \NAMEREF{implemented-values} )) 
    :  \TO \NAMEREF{implemented-records}\QUERY \\&\quad
    \leadsto \NAMEHYPER{../../../../../Funcons-beta/Values}{Value-Types}{when-true}
               (  \NAMEHYPER{../../../../../Funcons-beta/Values/Primitive}{Integers}{is-less-or-equal}
                       (  \NAMEHYPER{../../../../../Funcons-beta/Values/Composite}{Sequences}{length} \ 
                               \NAMEHYPER{../../../../../Funcons-beta/Values/Composite}{Maps}{map-elements} \ 
                                 \NAMEHYPER{../../../../../Funcons-beta/Values/Composite}{Records}{record-map} \ 
                                   \VAR{R}, 
                              4194303 ), 
                      \VAR{R} )
\end{align*}
In OCaml Light, records are non-mutable, and references are represented by
  mutable variables.

\begin{align*}
  \KEY{Type} \quad 
  & \NAMEDECL{implemented-references}  
    \leadsto \NAMEHYPER{../../../../../Funcons-beta/Computations/Normal}{Storing}{variables}
\end{align*}
\subsection{Arrays}\hypertarget{arrays}{}\label{arrays}

Arrays are finite, variable-sized sequences of values of the same type. 
  The current implementation supports arrays containing up to 2\^{}22 - 1 elements
  (4194303 elements) unless the elements are floating-point numbers (2097151
  elements in this case); (\ldots{})

\begin{align*}
  \KEY{Type} \quad 
  & \NAMEDECL{implemented-vectors} <: \NAMEHYPER{../../../../../Funcons-beta/Values/Composite}{Vectors}{vectors}
                                     (  \NAMEREF{implemented-values} ) \\&\quad
    \leadsto \NAMEHYPER{../../../../../Funcons-beta/Values/Composite}{Vectors}{vectors}
               (  \NAMEHYPER{../../../../../Funcons-beta/Values}{Value-Types}{values} )
\end{align*}
\begin{align*}
  \KEY{Funcon} \quad
  & \NAMEDECL{implemented-vector}(
                       \VAR{V} : \NAMEHYPER{../../../../../Funcons-beta/Values/Composite}{Vectors}{vectors}
                                 (  \NAMEREF{implemented-values} )) 
    :  \TO \NAMEREF{implemented-vectors}\QUERY \\&\quad
    \leadsto \NAMEHYPER{../../../../../Funcons-beta/Values}{Value-Types}{when-true}
               (  \NAMEHYPER{../../../../../Funcons-beta/Values/Primitive}{Integers}{is-less-or-equal}
                       (  \NAMEHYPER{../../../../../Funcons-beta/Values/Composite}{Sequences}{length} \ 
                               \NAMEHYPER{../../../../../Funcons-beta/Values/Composite}{Vectors}{vector-elements} \ 
                                 \VAR{V}, 
                              4194303 ), 
                      \VAR{V} )
\end{align*}
\subsection{Variant values}\hypertarget{variant-values}{}\label{variant-values}

Variant values are either a constant constructor, or a pair of a non-constant
  constructor and a value. The former case is written constr; the latter case
  is written (v1, \ldots{}, vn), where the vi are said to be the arguments of the 
  non-constant constructor constr. The parentheses may be omitted if there is
  only one argument. (\ldots{}) The current implementation limits each variant type
  to have at most 246 non-constant constructors and 2\^{}30-1 constant constructors.

\begin{align*}
  \KEY{Type} \quad 
  & \NAMEDECL{implemented-variants} <: \NAMEHYPER{../../../../../Funcons-beta/Values/Composite}{Variants}{variants}
                                     (  \NAMEREF{implemented-values} ) \\&\quad
    \leadsto \NAMEHYPER{../../../../../Funcons-beta/Values/Composite}{Variants}{variants}
               (  \NAMEHYPER{../../../../../Funcons-beta/Values}{Value-Types}{values} )
\end{align*}
\begin{align*}
  \KEY{Funcon} \quad
  & \NAMEDECL{implemented-variant}(
                       \VAR{V} : \NAMEHYPER{../../../../../Funcons-beta/Values/Composite}{Variants}{variants}
                                 (  \NAMEREF{implemented-values} )) 
    :  \TO \NAMEREF{implemented-variants} \\&\quad
    \leadsto \VAR{V}
\end{align*}
\subsection{Functions}\hypertarget{functions}{}\label{functions}

Functional values are mappings from values to values.

\begin{align*}
  \KEY{Type} \quad 
  & \NAMEDECL{implemented-functions} <: \NAMEHYPER{../../../../../Funcons-beta/Values/Abstraction}{Functions}{functions}
                                     (  \NAMEREF{implemented-values}, 
                                            \NAMEREF{implemented-values} ) \\&\quad
    \leadsto \NAMEHYPER{../../../../../Funcons-beta/Values/Abstraction}{Functions}{functions}
               (  \NAMEHYPER{../../../../../Funcons-beta/Values}{Value-Types}{values}, 
                      \NAMEHYPER{../../../../../Funcons-beta/Values}{Value-Types}{values} )
\end{align*}
\begin{align*}
  \KEY{Funcon} \quad
  & \NAMEDECL{implemented-function}(
                       \VAR{F} : \NAMEHYPER{../../../../../Funcons-beta/Values/Abstraction}{Functions}{functions}
                                 (  \NAMEREF{implemented-values}, 
                                        \NAMEREF{implemented-values} )) \\&\quad
    :  \TO \NAMEREF{implemented-functions} \\&\quad
    \leadsto \VAR{F}
\end{align*}
% 


