% 


\begin{center}
\rule{3in}{0.4pt}
\end{center}

\begin{displaymath}
\KEY{Language} \quad \STRING{MiniJava}
\end{displaymath}

{[}The MiniJava Reference Manual{]}: 
    \href{http://www.cambridge.org/us/features/052182060X/mjreference/mjreference.html}{http://www.cambridge.org/us/features/052182060X/mjreference/mjreference.html}

{[}Modern Compiler Implementation in Java: the MiniJava Project{]}:
    \href{http://www.cambridge.org/us/features/052182060X/}{http://www.cambridge.org/us/features/052182060X/}

The grammar used here is mostly a transliteration of the one provided at:
  \href{http://www.cambridge.org/us/features/052182060X/grammar.html}{http://www.cambridge.org/us/features/052182060X/grammar.html}
  (which differs in trivial ways from the one in the cited reference manual).

The rest of this file gives an overview of the MiniJava syntax. It is mostly
  in the form of a comment with embedded productions. The nonterminal symbols
  are hyperlinks to their actual specifications; similarly, section numbers
  (such as $\SECTHYPER{../.}{MiniJava-Dynamics}{1}$ below) link to the corresponding specification section.

$\SECTHYPER{../.}{MiniJava-Dynamics}{1}$ Programs

Syntax

\begin{align*}
  \VARDECL{START} : \SYNHYPER{../.}{MiniJava-Start}{start}
    \ ::= \ & \
    \SYNHYPER{../.}{MiniJava-Dynamics}{program}
  \\
  \VARDECL{P} : \SYNHYPER{../.}{MiniJava-Dynamics}{program}
    \ ::= \ & \
    \SYNHYPER{../.}{MiniJava-Dynamics}{main-class} \ \SYNHYPER{../.}{MiniJava-Dynamics}{class-declaration}\STAR
  \\
  \VARDECL{MC} : \SYNHYPER{../.}{MiniJava-Dynamics}{main-class}
    \ ::= \ & \
    \LEX{class} \ \SYNHYPER{../.}{MiniJava-Dynamics}{identifier} \ \LEX{{\LEFTBRACE}} \\
                                                             & \ \LEX{public} \ \LEX{static} \ \LEX{void} \ \LEX{main} \ \LEX{{(}} \ \LEX{String} \ \LEX{{[}} \ \LEX{{]}} \ \SYNHYPER{../.}{MiniJava-Dynamics}{identifier} \ \LEX{{)}} \ \LEX{{\LEFTBRACE}} \\
                                                                                                                                                                                                                                                                                                    & \ \SYNHYPER{../.}{MiniJava-Dynamics}{statement} \\
                                                                                                                                                                                                                                                                                                                         & \ \LEX{{\RIGHTBRACE}} \\
                                                                                                                                                                                                                                                                                                                                              & \ \LEX{{\RIGHTBRACE}}
\end{align*}
$\SECTHYPER{../.}{MiniJava-Dynamics}{2}$ Declarations

\begin{align*}
  \VARDECL{CD} : \SYNHYPER{../.}{MiniJava-Dynamics}{class-declaration}
    \ ::= \ & \
    \LEX{class} \ \SYNHYPER{../.}{MiniJava-Dynamics}{identifier} \ \LEFTGROUP \LEX{extends} \ \SYNHYPER{../.}{MiniJava-Dynamics}{identifier} \RIGHTGROUP\QUERY \ \LEX{{\LEFTBRACE}} \\
                                                                                  & \ \SYNHYPER{../.}{MiniJava-Dynamics}{var-declaration}\STAR \\
                                                                                                       & \ \SYNHYPER{../.}{MiniJava-Dynamics}{method-declaration}\STAR \\
                                                                                                                            & \ \LEX{{\RIGHTBRACE}}
  \\
  \VARDECL{VD} : \SYNHYPER{../.}{MiniJava-Dynamics}{var-declaration}
    \ ::= \ & \
    \SYNHYPER{../.}{MiniJava-Dynamics}{type} \ \SYNHYPER{../.}{MiniJava-Dynamics}{identifier} \ \LEX{{;}}
  \\
  \VARDECL{MD} : \SYNHYPER{../.}{MiniJava-Dynamics}{method-declaration}
    \ ::= \ & \
    \LEX{public} \ \SYNHYPER{../.}{MiniJava-Dynamics}{type} \ \SYNHYPER{../.}{MiniJava-Dynamics}{identifier} \ \LEX{{(}} \ \SYNHYPER{../.}{MiniJava-Dynamics}{formal-list}\QUERY \ \LEX{{)}} \ \LEX{{\LEFTBRACE}} \\
                                                                                                                                                 & \ \SYNHYPER{../.}{MiniJava-Dynamics}{var-declaration}\STAR \\
                                                                                                                                                                      & \ \SYNHYPER{../.}{MiniJava-Dynamics}{statement}\STAR \\
                                                                                                                                                                                           & \ \LEX{return} \ \SYNHYPER{../.}{MiniJava-Dynamics}{expression} \ \LEX{{;}} \\
                                                                                                                                                                                                                                                          & \ \LEX{{\RIGHTBRACE}}
  \\
  \VARDECL{T} : \SYNHYPER{../.}{MiniJava-Dynamics}{type}
    \ ::= \ & \
    \LEX{int} \ \LEX{{[}} \ \LEX{{]}} \\
    \ \mid \ & \ \LEX{boolean} \\
    \ \mid \ & \ \LEX{int} \\
    \ \mid \ & \ \SYNHYPER{../.}{MiniJava-Dynamics}{identifier}
  \\
  \VARDECL{FL} : \SYNHYPER{../.}{MiniJava-Dynamics}{formal-list}
    \ ::= \ & \
    \SYNHYPER{../.}{MiniJava-Dynamics}{type} \ \SYNHYPER{../.}{MiniJava-Dynamics}{identifier} \ \LEFTGROUP \LEX{{,}} \ \SYNHYPER{../.}{MiniJava-Dynamics}{formal-list} \RIGHTGROUP\QUERY
\end{align*}
$\SECTHYPER{../.}{MiniJava-Dynamics}{3}$ Statements

\begin{align*}
  \VARDECL{S} : \SYNHYPER{../.}{MiniJava-Dynamics}{statement}
    \ ::= \ & \
    \LEX{{\LEFTBRACE}} \ \SYNHYPER{../.}{MiniJava-Dynamics}{statement}\STAR \ \LEX{{\RIGHTBRACE}} \\
    \ \mid \ & \ \LEX{if} \ \LEX{{(}} \ \SYNHYPER{../.}{MiniJava-Dynamics}{expression} \ \LEX{{)}} \ \SYNHYPER{../.}{MiniJava-Dynamics}{statement} \ \LEX{else} \ \SYNHYPER{../.}{MiniJava-Dynamics}{statement} \\
    \ \mid \ & \ \LEX{while} \ \LEX{{(}} \ \SYNHYPER{../.}{MiniJava-Dynamics}{expression} \ \LEX{{)}} \ \SYNHYPER{../.}{MiniJava-Dynamics}{statement} \\
    \ \mid \ & \ \LEX{System} \ \LEX{{.}} \ \LEX{out} \ \LEX{{.}} \ \LEX{println} \ \LEX{{(}} \ \SYNHYPER{../.}{MiniJava-Dynamics}{expression} \ \LEX{{)}} \ \LEX{{;}} \\
    \ \mid \ & \ \SYNHYPER{../.}{MiniJava-Dynamics}{identifier} \ \LEX{{=}} \ \SYNHYPER{../.}{MiniJava-Dynamics}{expression} \ \LEX{{;}} \\
    \ \mid \ & \ \SYNHYPER{../.}{MiniJava-Dynamics}{identifier} \ \LEX{{[}} \ \SYNHYPER{../.}{MiniJava-Dynamics}{expression} \ \LEX{{]}} \ \LEX{{=}} \ \SYNHYPER{../.}{MiniJava-Dynamics}{expression} \ \LEX{{;}}
\end{align*}
$\SECTHYPER{../.}{MiniJava-Dynamics}{4}$ Expressions

\begin{align*}
  \VARDECL{E} : \SYNHYPER{../.}{MiniJava-Dynamics}{expression}
    \ ::= \ & \
    \SYNHYPER{../.}{MiniJava-Dynamics}{expression} \ \LEX{{\AMPERSAND}{\AMPERSAND}} \ \SYNHYPER{../.}{MiniJava-Dynamics}{expression} \\
    \ \mid \ & \ \SYNHYPER{../.}{MiniJava-Dynamics}{expression} \ \LEX{{<}} \ \SYNHYPER{../.}{MiniJava-Dynamics}{expression} \\
    \ \mid \ & \ \SYNHYPER{../.}{MiniJava-Dynamics}{expression} \ \LEX{{+}} \ \SYNHYPER{../.}{MiniJava-Dynamics}{expression} \\
    \ \mid \ & \ \SYNHYPER{../.}{MiniJava-Dynamics}{expression} \ \LEX{{-}} \ \SYNHYPER{../.}{MiniJava-Dynamics}{expression} \\
    \ \mid \ & \ \SYNHYPER{../.}{MiniJava-Dynamics}{expression} \ \LEX{{*}} \ \SYNHYPER{../.}{MiniJava-Dynamics}{expression} \\
    \ \mid \ & \ \SYNHYPER{../.}{MiniJava-Dynamics}{expression} \ \LEX{{[}} \ \SYNHYPER{../.}{MiniJava-Dynamics}{expression} \ \LEX{{]}} \\
    \ \mid \ & \ \SYNHYPER{../.}{MiniJava-Dynamics}{expression} \ \LEX{{.}} \ \LEX{length} \\
    \ \mid \ & \ \SYNHYPER{../.}{MiniJava-Dynamics}{expression} \ \LEX{{.}} \ \SYNHYPER{../.}{MiniJava-Dynamics}{identifier} \ \LEX{{(}} \ \SYNHYPER{../.}{MiniJava-Dynamics}{expression-list}\QUERY \ \LEX{{)}} \\
    \ \mid \ & \ \SYNHYPER{../.}{MiniJava-Dynamics}{integer-literal} \\
    \ \mid \ & \ \LEX{true} \\
    \ \mid \ & \ \LEX{false} \\
    \ \mid \ & \ \SYNHYPER{../.}{MiniJava-Dynamics}{identifier} \\
    \ \mid \ & \ \LEX{this} \\
    \ \mid \ & \ \LEX{new} \ \LEX{int} \ \LEX{{[}} \ \SYNHYPER{../.}{MiniJava-Dynamics}{expression} \ \LEX{{]}} \\
    \ \mid \ & \ \LEX{new} \ \SYNHYPER{../.}{MiniJava-Dynamics}{identifier} \ \LEX{{(}} \ \LEX{{)}} \\
    \ \mid \ & \ \LEX{{!}} \ \SYNHYPER{../.}{MiniJava-Dynamics}{expression} \\
    \ \mid \ & \ \LEX{{(}} \ \SYNHYPER{../.}{MiniJava-Dynamics}{expression} \ \LEX{{)}}
  \\
  \VARDECL{EL} : \SYNHYPER{../.}{MiniJava-Dynamics}{expression-list}
    \ ::= \ & \
    \SYNHYPER{../.}{MiniJava-Dynamics}{expression} \ \LEFTGROUP \LEX{{,}} \ \SYNHYPER{../.}{MiniJava-Dynamics}{expression-list} \RIGHTGROUP\QUERY
\end{align*}
$\SECTHYPER{../.}{MiniJava-Dynamics}{5}$ Lexemes

Lexis

\begin{align*}
  \VARDECL{ID} : \SYNHYPER{../.}{MiniJava-Dynamics}{identifier}
    \ ::= \ & \
    \SYNHYPER{../.}{MiniJava-Dynamics}{letter} \ \LEFTGROUP \SYNHYPER{../.}{MiniJava-Dynamics}{letter} \mid \SYNHYPER{../.}{MiniJava-Dynamics}{digit} \mid \LEX{{\UNDERSCORE}} \RIGHTGROUP\STAR
  \\
  \VARDECL{IL} : \SYNHYPER{../.}{MiniJava-Dynamics}{integer-literal}
    \ ::= \ & \
    \SYNHYPER{../.}{MiniJava-Dynamics}{digit}\PLUS
  \\
   \SYNHYPER{../.}{MiniJava-Dynamics}{letter}
    \ ::= \ & \
    \LEX{a} {-} \LEX{z} \mid \LEX{A} {-} \LEX{Z}
  \\
   \SYNHYPER{../.}{MiniJava-Dynamics}{digit}
    \ ::= \ & \
    \LEX{0} {-} \LEX{9}
\end{align*}
\section{$\SECT{6}$ Disambiguation}\hypertarget{sect6-disambiguation}{}\label{sect6-disambiguation}

The mixture of CBS and SDF below specifies how MiniJava texts are to
  be disambiguated by parsers generated from the above grammar.

The specified rules are adequate to disambiguate all the example programs
  provided at http://www.cambridge.org/us/features/052182060X/\#progs

$\KEY{Syntax SDF}$

\begin{quote}
context-free syntax\newline
   $\SHADE{\quad\SYNHYPER{../.}{MiniJava-Dynamics}{expression}  \ ::= \  \  \SYNHYPER{../.}{MiniJava-Dynamics}{expression} \ \LEX{{*}} \ \SYNHYPER{../.}{MiniJava-Dynamics}{expression}}$ \{left\}\newline
   $\SHADE{\quad\SYNHYPER{../.}{MiniJava-Dynamics}{expression}  \ ::= \  \  \SYNHYPER{../.}{MiniJava-Dynamics}{expression} \ \LEX{{+}} \ \SYNHYPER{../.}{MiniJava-Dynamics}{expression}}$ \{left\}\newline
   $\SHADE{\quad\SYNHYPER{../.}{MiniJava-Dynamics}{expression}  \ ::= \  \  \SYNHYPER{../.}{MiniJava-Dynamics}{expression} \ \LEX{{-}} \ \SYNHYPER{../.}{MiniJava-Dynamics}{expression}}$ \{left\}\newline
   $\SHADE{\quad\SYNHYPER{../.}{MiniJava-Dynamics}{expression}  \ ::= \  \  \SYNHYPER{../.}{MiniJava-Dynamics}{expression} \ \LEX{{<}} \ \SYNHYPER{../.}{MiniJava-Dynamics}{expression}}$ \{non-assoc\}\newline
   $\SHADE{\quad\SYNHYPER{../.}{MiniJava-Dynamics}{expression}  \ ::= \  \  \SYNHYPER{../.}{MiniJava-Dynamics}{expression} \ \LEX{{\AMPERSAND}{\AMPERSAND}} \ \SYNHYPER{../.}{MiniJava-Dynamics}{expression}}$ \{left\}\newline
   \newline
   context-free priorities\newline
   \{\newline
   $\SHADE{\quad\SYNHYPER{../.}{MiniJava-Dynamics}{expression}  \ ::= \  \  \SYNHYPER{../.}{MiniJava-Dynamics}{expression} \ \LEX{{.}} \ \SYNHYPER{../.}{MiniJava-Dynamics}{identifier} \ \LEX{{(}} \ \SYNHYPER{../.}{MiniJava-Dynamics}{expression-list}\QUERY \ \LEX{{)}}}$\newline
   $\SHADE{\quad\SYNHYPER{../.}{MiniJava-Dynamics}{expression}  \ ::= \  \  \SYNHYPER{../.}{MiniJava-Dynamics}{expression} \ \LEX{{[}} \ \SYNHYPER{../.}{MiniJava-Dynamics}{expression} \ \LEX{{]}}}$\newline
   \} \textless{}0\textgreater{} \textgreater{}\newline
   $\SHADE{\quad\SYNHYPER{../.}{MiniJava-Dynamics}{expression}  \ ::= \  \  \LEX{{!}} \ \SYNHYPER{../.}{MiniJava-Dynamics}{expression}}$\newline
   \textgreater{}\newline
   $\SHADE{\quad\SYNHYPER{../.}{MiniJava-Dynamics}{expression}  \ ::= \  \  \SYNHYPER{../.}{MiniJava-Dynamics}{expression} \ \LEX{{*}} \ \SYNHYPER{../.}{MiniJava-Dynamics}{expression}}$\newline
   \textgreater{} \{\newline
   $\SHADE{\quad\SYNHYPER{../.}{MiniJava-Dynamics}{expression}  \ ::= \  \  \SYNHYPER{../.}{MiniJava-Dynamics}{expression} \ \LEX{{+}} \ \SYNHYPER{../.}{MiniJava-Dynamics}{expression}}$\newline
   $\SHADE{\quad\SYNHYPER{../.}{MiniJava-Dynamics}{expression}  \ ::= \  \  \SYNHYPER{../.}{MiniJava-Dynamics}{expression} \ \LEX{{-}} \ \SYNHYPER{../.}{MiniJava-Dynamics}{expression}}$\newline
   \} \textgreater{}\newline
   $\SHADE{\quad\SYNHYPER{../.}{MiniJava-Dynamics}{expression}  \ ::= \  \  \SYNHYPER{../.}{MiniJava-Dynamics}{expression} \ \LEX{{<}} \ \SYNHYPER{../.}{MiniJava-Dynamics}{expression}}$\newline
   \textgreater{}\newline
   $\SHADE{\quad\SYNHYPER{../.}{MiniJava-Dynamics}{expression}  \ ::= \  \  \SYNHYPER{../.}{MiniJava-Dynamics}{expression} \ \LEX{{\AMPERSAND}{\AMPERSAND}} \ \SYNHYPER{../.}{MiniJava-Dynamics}{expression}}$
\end{quote}

$\KEY{Lexis SDF}$

\begin{quote}
lexical restrictions\newline
   $\SHADE{\SYNHYPER{../.}{MiniJava-Dynamics}{identifier}}$      -/- {[}a-zA-Z0-9\_{]}\newline
   $\SHADE{\SYNHYPER{../.}{MiniJava-Dynamics}{integer-literal}}$ -/- {[}0-9{]}\newline
   \newline
   lexical syntax\newline
   $\SHADE{\SYNHYPER{../.}{MiniJava-Dynamics}{identifier}}$ = $\SHADE{\SYNREF{reserved-id}}$ \{reject\}
\end{quote}

\begin{align*}
  \KEY{Lexis} \quad
     \SYN{reserved-id}
      \ ::= \ & \
      \LEX{String} \\
      \ \mid \ & \ \LEX{System} \\
      \ \mid \ & \ \LEX{boolean} \\
      \ \mid \ & \ \LEX{class} \\
      \ \mid \ & \ \LEX{else} \\
      \ \mid \ & \ \LEX{extends} \\
      \ \mid \ & \ \LEX{false} \\
      \ \mid \ & \ \LEX{if} \\
      \ \mid \ & \ \LEX{int} \\
      \ \mid \ & \ \LEX{length} \\
      \ \mid \ & \ \LEX{main} \\
      \ \mid \ & \ \LEX{new} \\
      \ \mid \ & \ \LEX{out} \\
      \ \mid \ & \ \LEX{println} \\
      \ \mid \ & \ \LEX{public} \\
      \ \mid \ & \ \LEX{return} \\
      \ \mid \ & \ \LEX{static} \\
      \ \mid \ & \ \LEX{this} \\
      \ \mid \ & \ \LEX{true} \\
      \ \mid \ & \ \LEX{void}
\end{align*}
% 


