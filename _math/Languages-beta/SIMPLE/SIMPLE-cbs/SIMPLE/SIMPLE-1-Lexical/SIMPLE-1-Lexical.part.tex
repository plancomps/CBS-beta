\begin{displaymath}
\KEY{Language} \STRING{SIMPLE}
\end{displaymath}

\section*{$\SECT{1}$ Lexical Syntax}\hypertarget{sect1-lexical-syntax}{}\label{sect1-lexical-syntax}

\begin{align*}
  \KEY{Syntax} ~ 
    \VARDECL{V} : \SYNDECL{value}
      ~ ::= ~ &
      \SYNREF{bool} \\
      ~ \mid ~ &  \SYNREF{int} \\
      ~ \mid ~ &  \SYNREF{string}
    \\
     \SYNDECL{bool}
      ~ ::= ~ &
      \LEX{false} \\
      ~ \mid ~ &  \LEX{true}
\\
  \KEY{Lexis} ~ 
    \VARDECL{Z} : \SYN{int}
      ~ ::= ~ & \LEFTGROUP \LEX{0} - \LEX{9} \RIGHTGROUP\PLUS
\\
  \KEY{Syntax} ~ 
     \SYN{string}
      ~ ::= ~ & \LEX{"{}} \_ \SYNREF{string-chars} \_ \LEX{"{}}
\\
  \KEY{Lexis} ~ 
    \VARDECL{S} : \SYN{string-chars}
      ~ ::= ~ & \LEFTGROUP \mathop{\sim} \LEX{"{}} \RIGHTGROUP\STAR
\end{align*}
\begin{align*}
  \KEY{Semantics} ~ 
  & \SEMDECL{val} \LEFTPHRASE ~ \_ : \SYNREF{value} ~ \RIGHTPHRASE  
    :  \TO \NAMEHYPER{../../../../../Funcons-beta/Values}{Value-Types}{values}
\end{align*}
\subsection*{$\SECT{1.1}$ Booleans}\hypertarget{sect11-booleans}{}\label{sect11-booleans}

\begin{align*}
  \KEY{Rule} ~ 
    & \SEMREF{val} \LEFTPHRASE ~ \LEX{false} ~ \RIGHTPHRASE  = \\&\quad
      \NAMEHYPER{../../../../../Funcons-beta/Values/Primitive}{Booleans}{false}
\\
  \KEY{Rule} ~ 
    & \SEMREF{val} \LEFTPHRASE ~ \LEX{true} ~ \RIGHTPHRASE  = \\&\quad
      \NAMEHYPER{../../../../../Funcons-beta/Values/Primitive}{Booleans}{true}
\end{align*}
\subsection*{$\SECT{1.2}$ Integers}\hypertarget{sect12-integers}{}\label{sect12-integers}

\begin{align*}
  \KEY{Rule} ~ 
    & \SEMREF{val} \LEFTPHRASE ~ \VARREF{Z} ~ \RIGHTPHRASE  = \\&\quad
      \NAMEHYPER{../../../../../Funcons-beta/Values/Primitive}{Integers}{decimal-natural}
        ( \textsf{\textquotedblleft}\VAR{Z}\textsf{\textquotedblright} )
\end{align*}
\subsection*{$\SECT{1.3}$ Strings}\hypertarget{sect13-strings}{}\label{sect13-strings}

\begin{align*}
  \KEY{Rule} ~ 
    & \SEMREF{val} \LEFTPHRASE ~ \LEX{"{}} ~ \VARREF{S} ~ \LEX{"{}} ~ \RIGHTPHRASE  = \\&\quad
      \textsf{\textquotedblleft}\VAR{S}\textsf{\textquotedblright}
\end{align*}
\subsection*{$\SECT{1.4}$ Identifiers}\hypertarget{sect14-identifiers}{}\label{sect14-identifiers}

\begin{align*}
  \KEY{Lexis} ~ 
    \VARDECL{Id} : \SYN{id}
      ~ ::= ~ & \LEFTGROUP \LEX{A} - \LEX{Z} \mid \LEX{a} - \LEX{z} \mid \LEX{\UNDERSCORE } \RIGHTGROUP ~ \LEFTGROUP \LEX{0} - \LEX{9} \mid \LEX{A} - \LEX{Z} \mid \LEX{a} - \LEX{z} \mid \LEX{\UNDERSCORE } \RIGHTGROUP\STAR
\end{align*}
\begin{align*}
  \KEY{Semantics} ~ 
  & \SEMDECL{id} \LEFTPHRASE ~ \_ : \SYNREF{id} ~ \RIGHTPHRASE  
    :  \TO \NAMEHYPER{../../../../../Funcons-beta/Computations/Normal}{Binding}{ids}
\\
  \KEY{Rule} ~ 
    & \SEMREF{id} \LEFTPHRASE ~ \VARREF{Id} ~ \RIGHTPHRASE  = \\&\quad
      \textsf{\textquotedblleft}\VAR{Id}\textsf{\textquotedblright}
\end{align*}
\begin{align*}
  \KEY{Lexis} ~ 
     \SYNDECL{keyword}
      ~ ::= ~ &
      \LEX{false} \\
      ~ \mid ~ &  \LEX{true} \\
      ~ \mid ~ &  \LEX{var} \\
      ~ \mid ~ &  \LEX{function} \\
      ~ \mid ~ &  \LEX{sizeOf} \\
      ~ \mid ~ &  \LEX{read} \\
      ~ \mid ~ &  \LEX{spawn} \\
      ~ \mid ~ &  \LEX{if} \\
      ~ \mid ~ &  \LEX{else} \\
      ~ \mid ~ &  \LEX{while} \\
      ~ \mid ~ &  \LEX{for} \\
      ~ \mid ~ &  \LEX{print} \\
      ~ \mid ~ &  \LEX{return} \\
      ~ \mid ~ &  \LEX{try} \\
      ~ \mid ~ &  \LEX{catch} \\
      ~ \mid ~ &  \LEX{throw} \\
      ~ \mid ~ &  \LEX{join} \\
      ~ \mid ~ &  \LEX{acquire} \\
      ~ \mid ~ &  \LEX{release} \\
      ~ \mid ~ &  \LEX{rendezvous}
\end{align*}
