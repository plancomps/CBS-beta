\subsubsection*{Thunks}\hypertarget{thunks}{}\label{thunks}

\begin{align*}
  [ ~ 
  \KEY{Datatype} ~ & \NAMEREF{thunks} \\
  \KEY{Funcon} ~ & \NAMEREF{thunk} \\
  \KEY{Funcon} ~ & \NAMEREF{force}
  ~ ]
\end{align*}
\begin{align*}
  \KEY{Meta-variables} ~ 
  & \VAR{T} <: \NAMEHYPER{../..}{Value-Types}{values}
\end{align*}
\begin{align*}
  \KEY{Datatype} ~ 
  \NAMEDECL{thunks}(\VAR{T} )  
  ~ ::= ~ & \NAMEDECL{thunk} (\_ : \NAMEHYPER{../.}{Generic}{abstractions}
                                         ( (  ~  ) \TO \VAR{T} ))
\end{align*}
$\SHADE{\NAMEREF{thunks}
           ( \VAR{T} )}$ consists of abstractions whose bodies do not depend on
  a given value, and whose executions normally compute values of type $\SHADE{\VAR{T}}$.
  $\SHADE{\NAMEREF{thunk}
           ( \NAMEHYPER{../.}{Generic}{abstraction}
               ( \VAR{X} ) )}$ evaluates to a thunk with dynamic bindings,
  $\SHADE{\NAMEREF{thunk}
           ( \NAMEHYPER{../.}{Generic}{closure}
               ( \VAR{X} ) )}$ computes a thunk with static bindings.

\begin{align*}
  \KEY{Funcon} ~ 
  & \NAMEDECL{force}(\_ : \NAMEREF{thunks}
                                ( \VAR{T} )) :  \TO \VAR{T}
\end{align*}
$\SHADE{\NAMEREF{force}
           ( \VAR{H} )}$ enacts the abstraction of the thunk $\SHADE{\VAR{H}}$.

\begin{align*}
  \KEY{Rule} ~ 
    & \NAMEREF{force}
        ( \NAMEREF{thunk}
            ( \NAMEHYPER{../.}{Generic}{abstraction}
                ( \VAR{X} ) ) ) \leadsto
        \NAMEHYPER{../../../Computations/Normal}{Giving}{no-given}
          ( \VAR{X} )
\end{align*}
