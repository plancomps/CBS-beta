% 


\begin{center}
\rule{3in}{0.4pt}
\end{center}

\subsubsection{Trees}\hypertarget{trees}{}\label{trees}

\begin{align*}
  [ \
  \KEY{Datatype} \quad & \NAMEREF{trees} \\
  \KEY{Funcon} \quad & \NAMEREF{tree} \\
  \KEY{Funcon} \quad & \NAMEREF{tree-root-value} \\
  \KEY{Funcon} \quad & \NAMEREF{tree-branch-sequence} \\
  \KEY{Funcon} \quad & \NAMEREF{single-branching-sequence} \\
  \KEY{Funcon} \quad & \NAMEREF{forest-root-value-sequence} \\
  \KEY{Funcon} \quad & \NAMEREF{forest-branch-sequence} \\
  \KEY{Funcon} \quad & \NAMEREF{forest-value-sequence}
  \ ]
\end{align*}
\begin{align*}
  \KEY{Meta-variables} \quad
  & \VAR{T} <: \NAMEHYPER{../..}{Value-Types}{values}
\end{align*}
\begin{align*}
  \KEY{Datatype} \quad 
  \NAMEDECL{trees}(
                     \VAR{T} ) 
  \ ::= \ & \NAMEDECL{tree}(
                               \_ : \VAR{T}, \_ : (  \NAMEREF{trees}
                                               (  \VAR{T} ) )\STAR)
\end{align*}
$\SHADE{\NAMEREF{trees}
           (  \VAR{T} )}$ consists of finitely-branching trees with elements of type $\SHADE{\VAR{T}}$.
  When $\SHADE{\VAR{V} : \VAR{T}}$, $\SHADE{\NAMEREF{tree}
           (  \VAR{V} )}$ is a leaf, and $\SHADE{\NAMEREF{tree}
           (  \VAR{V}, 
                  \VAR{B}\SUB{1}, 
                  \cdots, 
                  \VAR{B}\SUB{n} )}$ is a tree with
  branches $\SHADE{\VAR{B}\SUB{1}}$, \ldots{}, $\SHADE{\VAR{B}\SUB{n}}$.

\begin{align*}
  \KEY{Funcon} \quad
  & \NAMEDECL{tree-root-value}(
                       \_ : \NAMEREF{trees}
                                 (  \VAR{T} )) 
    :  \TO (  \VAR{T} )\QUERY 
\\
  \KEY{Rule} \quad
    & \NAMEREF{tree-root-value} \ 
        \NAMEREF{tree}
          (  \VAR{V} : \VAR{T}, 
                 \_\STAR : (  \NAMEREF{trees}
                                  (  \VAR{T} ) )\STAR ) \leadsto 
        \VAR{V}
\end{align*}
\begin{align*}
  \KEY{Funcon} \quad
  & \NAMEDECL{tree-branch-sequence}(
                       \_ : \NAMEREF{trees}
                                 (  \VAR{T} )) 
    :  \TO (  \NAMEREF{trees}
                           (  \VAR{T} ) )\STAR 
\\
  \KEY{Rule} \quad
    & \NAMEREF{tree-branch-sequence} \ 
        \NAMEREF{tree}
          (  \_ : \VAR{T}, 
                 \VAR{B}\STAR : (  \NAMEREF{trees}
                                  (  \VAR{T} ) )\STAR ) \leadsto 
        \VAR{B}\STAR
\end{align*}
\begin{align*}
  \KEY{Funcon} \quad
  & \NAMEDECL{single-branching-sequence}(
                       \_ : \NAMEREF{trees}
                                 (  \VAR{T} )) 
    :  \TO \VAR{T}\PLUS 
\end{align*}
$\SHADE{\NAMEREF{single-branching-sequence} \ 
           \VAR{B}}$ extracts the values in $\SHADE{\VAR{B}}$ starting from 
  the root, provided that $\SHADE{\VAR{B}}$ is at most single-branching; otherwise it fails.

\begin{align*}
  \KEY{Rule} \quad
    & \NAMEREF{single-branching-sequence} \ 
        \NAMEREF{tree}
          (  \VAR{V} : \VAR{T} ) \leadsto 
        \VAR{V}
\\
  \KEY{Rule} \quad
    & \NAMEREF{single-branching-sequence} \ 
        \NAMEREF{tree}
          (  \VAR{V} : \VAR{T}, 
                 \VAR{B} : \NAMEREF{trees}
                            (  \VAR{T} ) ) \leadsto \\&\quad
        \NAMEHYPER{../../../Computations/Normal}{Flowing}{left-to-right}
          (  \VAR{V}, 
                 \NAMEREF{single-branching-sequence} \ 
                  \VAR{B} )
\\
  \KEY{Rule} \quad
    & \NAMEREF{single-branching-sequence} \ 
        \NAMEREF{tree}
          (  \_ : \VAR{T}, 
                 \_ : \NAMEREF{trees}
                            (  \VAR{T} ), 
                 \_\PLUS : (  \NAMEREF{trees}
                                  (  \VAR{T} ) )\PLUS ) \leadsto 
        \NAMEHYPER{../../../Computations/Abnormal}{Failing}{fail}
\end{align*}
A sequence of trees corresponds to a forest, and the selector funcons
  on trees $\SHADE{\VAR{B}}$ extend to forests $\SHADE{\VAR{B}\STAR}$:

\begin{align*}
  \KEY{Funcon} \quad
  & \NAMEDECL{forest-root-value-sequence}(
                       \_ : (  \NAMEREF{trees}
                                       (  \VAR{T} ) )\STAR) 
    :  \TO \VAR{T}\STAR 
\\
  \KEY{Rule} \quad
    & \NAMEREF{forest-root-value-sequence}
        (  \VAR{B} : \NAMEREF{trees}
                          (  \VAR{T} ), 
               \VAR{B}\STAR : (  \NAMEREF{trees}
                                (  \VAR{T} ) )\STAR ) \leadsto \\&\quad
        (  \NAMEREF{tree-root-value} \ 
                \VAR{B}, 
               \NAMEREF{forest-root-value-sequence} \ 
                \VAR{B}\STAR )
\\
  \KEY{Rule} \quad
    & \NAMEREF{forest-root-value-sequence}
        (   \  ) \leadsto 
        (   \  )
\end{align*}
\begin{align*}
  \KEY{Funcon} \quad
  & \NAMEDECL{forest-branch-sequence}(
                       \_ : (  \NAMEREF{trees}
                                       (  \VAR{T} ) )\STAR) 
    :  \TO \VAR{T}\STAR 
\\
  \KEY{Rule} \quad
    & \NAMEREF{forest-branch-sequence}
        (  \VAR{B} : \NAMEREF{trees}
                          (  \VAR{T} ), 
               \VAR{B}\STAR : (  \NAMEREF{trees}
                                (  \VAR{T} ) )\STAR ) \leadsto \\&\quad
        (  \NAMEREF{tree-branch-sequence} \ 
                \VAR{B}, 
               \NAMEREF{forest-branch-sequence} \ 
                \VAR{B}\STAR )
\\
  \KEY{Rule} \quad
    & \NAMEREF{forest-branch-sequence}
        (   \  ) \leadsto 
        (   \  )
\end{align*}
\begin{align*}
  \KEY{Funcon} \quad
  & \NAMEDECL{forest-value-sequence}(
                       \_ : (  \NAMEREF{trees}
                                       (  \VAR{T} ) )\STAR) 
    :  \TO \VAR{T}\STAR 
\end{align*}
$\SHADE{\NAMEREF{forest-value-sequence} \ 
           \VAR{B}\STAR}$ provides the values from a left-to-right pre-order
  depth-first traversal.

\begin{align*}
  \KEY{Rule} \quad
    & \NAMEREF{forest-value-sequence}
        (  \NAMEREF{tree}
                (  \VAR{V} : \VAR{T}, 
                       \VAR{B}\SUB{1}\STAR : (  \NAMEREF{trees}
                                        (  \VAR{T} ) )\STAR ), 
               \VAR{B}\SUB{2}\STAR : (  \NAMEREF{trees}
                                (  \VAR{T} ) )\STAR ) \leadsto \\&\quad
        (  \VAR{V}, 
               \NAMEREF{forest-value-sequence} \ 
                \VAR{B}\SUB{1}\STAR, 
               \NAMEREF{forest-value-sequence} \ 
                \VAR{B}\SUB{2}\STAR )
\\
  \KEY{Rule} \quad
    & \NAMEREF{forest-value-sequence}
        (   \  ) \leadsto 
        (   \  )
\end{align*}
Other linearizations of trees can be added: breadth-first, right-to-left,
  C3, etc.

% 


