\subsubsection*{Trees}\hypertarget{trees}{}\label{trees}

\begin{align*}
  [ ~ 
  \KEY{Datatype} ~ & \NAMEREF{trees} \\
  \KEY{Funcon} ~ & \NAMEREF{tree} \\
  \KEY{Funcon} ~ & \NAMEREF{tree-root-value} \\
  \KEY{Funcon} ~ & \NAMEREF{tree-branch-sequence} \\
  \KEY{Funcon} ~ & \NAMEREF{single-branching-sequence} \\
  \KEY{Funcon} ~ & \NAMEREF{forest-root-value-sequence} \\
  \KEY{Funcon} ~ & \NAMEREF{forest-branch-sequence} \\
  \KEY{Funcon} ~ & \NAMEREF{forest-value-sequence}
  ~ ]
\end{align*}
\begin{align*}
  \KEY{Meta-variables} ~ 
  & \VAR{T} <: \NAMEHYPER{../..}{Value-Types}{values}
\end{align*}
\begin{align*}
  \KEY{Datatype} ~ 
  \NAMEDECL{trees}(\VAR{T} )  
  ~ ::= ~ & \NAMEDECL{tree} (\_ : \VAR{T}, \_ : ( \NAMEREF{trees}
                                           ( \VAR{T} ) )\STAR)
\end{align*}
$\SHADE{\NAMEREF{trees}
           ( \VAR{T} )}$ consists of finitely-branching trees with elements of type $\SHADE{\VAR{T}}$.
  When $\SHADE{\VAR{V} : \VAR{T}}$, $\SHADE{\NAMEREF{tree}
           ( \VAR{V} )}$ is a leaf, and $\SHADE{\NAMEREF{tree}
           ( \VAR{V},   
             \VAR{B}\SUB{1},   
             \cdots,   
             \VAR{B}\SUB{n} )}$ is a tree with
  branches $\SHADE{\VAR{B}\SUB{1}}$, \ldots{}, $\SHADE{\VAR{B}\SUB{n}}$.

\begin{align*}
  \KEY{Funcon} ~ 
  & \NAMEDECL{tree-root-value}(\_ : \NAMEREF{trees}
                                ( \VAR{T} )) :  \TO ( \VAR{T} )\QUERY
\\
  \KEY{Rule} ~ 
    & \NAMEREF{tree-root-value} ~
        \NAMEREF{tree}
          ( \VAR{V} : \VAR{T},    
            \_\STAR : ( \NAMEREF{trees}
                          ( \VAR{T} ) )\STAR ) \leadsto
        \VAR{V}
\end{align*}
\begin{align*}
  \KEY{Funcon} ~ 
  & \NAMEDECL{tree-branch-sequence}(\_ : \NAMEREF{trees}
                                ( \VAR{T} )) :  \TO ( \NAMEREF{trees}
                                                                           ( \VAR{T} ) )\STAR
\\
  \KEY{Rule} ~ 
    & \NAMEREF{tree-branch-sequence} ~
        \NAMEREF{tree}
          ( \_ : \VAR{T},    
            \VAR{B}\STAR : ( \NAMEREF{trees}
                          ( \VAR{T} ) )\STAR ) \leadsto
        \VAR{B}\STAR
\end{align*}
\begin{align*}
  \KEY{Funcon} ~ 
  & \NAMEDECL{single-branching-sequence}(\_ : \NAMEREF{trees}
                                ( \VAR{T} )) :  \TO \VAR{T}\PLUS
\end{align*}
$\SHADE{\NAMEREF{single-branching-sequence} ~
           \VAR{B}}$ extracts the values in $\SHADE{\VAR{B}}$ starting from 
  the root, provided that $\SHADE{\VAR{B}}$ is at most single-branching; otherwise it fails.

\begin{align*}
  \KEY{Rule} ~ 
    & \NAMEREF{single-branching-sequence} ~
        \NAMEREF{tree}
          ( \VAR{V} : \VAR{T} ) \leadsto
        \VAR{V}
\\
  \KEY{Rule} ~ 
    & \NAMEREF{single-branching-sequence} ~
        \NAMEREF{tree}
          ( \VAR{V} : \VAR{T},    
            \VAR{B} : \NAMEREF{trees}
                        ( \VAR{T} ) ) \leadsto
        \NAMEHYPER{../../../Computations/Normal}{Flowing}{left-to-right}
          ( \VAR{V},   
            \NAMEREF{single-branching-sequence} ~
              \VAR{B} )
\\
  \KEY{Rule} ~ 
    & \NAMEREF{single-branching-sequence} ~
        \NAMEREF{tree}
          ( \_ : \VAR{T},    
            \_ : \NAMEREF{trees}
                        ( \VAR{T} ),    
            \_\PLUS : ( \NAMEREF{trees}
                          ( \VAR{T} ) )\PLUS ) \leadsto
        \NAMEHYPER{../../../Computations/Abnormal}{Failing}{fail}
\end{align*}
A sequence of trees corresponds to a forest, and the selector funcons
  on trees $\SHADE{\VAR{B}}$ extend to forests $\SHADE{\VAR{B}\STAR}$:

\begin{align*}
  \KEY{Funcon} ~ 
  & \NAMEDECL{forest-root-value-sequence}(\_ : ( \NAMEREF{trees}
                                  ( \VAR{T} ) )\STAR) :  \TO \VAR{T}\STAR
\\
  \KEY{Rule} ~ 
    & \NAMEREF{forest-root-value-sequence}
        ( \VAR{B} : \NAMEREF{trees}
                      ( \VAR{T} ),   
          \VAR{B}\STAR : ( \NAMEREF{trees}
                        ( \VAR{T} ) )\STAR ) \leadsto
        ( \NAMEREF{tree-root-value} ~
            \VAR{B},  
          \NAMEREF{forest-root-value-sequence} ~
            \VAR{B}\STAR )
\\
  \KEY{Rule} ~ 
    & \NAMEREF{forest-root-value-sequence}
        (  ~  ) \leadsto
        (  ~  )
\end{align*}
\begin{align*}
  \KEY{Funcon} ~ 
  & \NAMEDECL{forest-branch-sequence}(\_ : ( \NAMEREF{trees}
                                  ( \VAR{T} ) )\STAR) :  \TO \VAR{T}\STAR
\\
  \KEY{Rule} ~ 
    & \NAMEREF{forest-branch-sequence}
        ( \VAR{B} : \NAMEREF{trees}
                      ( \VAR{T} ),   
          \VAR{B}\STAR : ( \NAMEREF{trees}
                        ( \VAR{T} ) )\STAR ) \leadsto
        ( \NAMEREF{tree-branch-sequence} ~
            \VAR{B},  
          \NAMEREF{forest-branch-sequence} ~
            \VAR{B}\STAR )
\\
  \KEY{Rule} ~ 
    & \NAMEREF{forest-branch-sequence}
        (  ~  ) \leadsto
        (  ~  )
\end{align*}
\begin{align*}
  \KEY{Funcon} ~ 
  & \NAMEDECL{forest-value-sequence}(\_ : ( \NAMEREF{trees}
                                  ( \VAR{T} ) )\STAR) :  \TO \VAR{T}\STAR
\end{align*}
$\SHADE{\NAMEREF{forest-value-sequence} ~
           \VAR{B}\STAR}$ provides the values from a left-to-right pre-order
  depth-first traversal.

\begin{align*}
  \KEY{Rule} ~ 
    & \NAMEREF{forest-value-sequence}
        ( \NAMEREF{tree}
            ( \VAR{V} : \VAR{T},    
              \VAR{B}\SUB{1}\STAR : ( \NAMEREF{trees}
                            ( \VAR{T} ) )\STAR ),   
          \VAR{B}\SUB{2}\STAR : ( \NAMEREF{trees}
                        ( \VAR{T} ) )\STAR ) \leadsto
        ( \VAR{V},  
          \NAMEREF{forest-value-sequence} ~
            \VAR{B}\SUB{1}\STAR,  
          \NAMEREF{forest-value-sequence} ~
            \VAR{B}\SUB{2}\STAR )
\\
  \KEY{Rule} ~ 
    & \NAMEREF{forest-value-sequence}
        (  ~  ) \leadsto
        (  ~  )
\end{align*}
Other linearizations of trees can be added: breadth-first, right-to-left,
  C3, etc.

