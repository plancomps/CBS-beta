\subsubsection*{Tuples}\hypertarget{tuples}{}\label{tuples}

\begin{align*}
  [ ~ 
  \KEY{Datatype} ~ & \NAMEREF{tuples} \\
  \KEY{Funcon} ~ & \NAMEREF{tuple-elements} \\
  \KEY{Funcon} ~ & \NAMEREF{tuple-zip}
  ~ ]
\end{align*}
\begin{align*}
  \KEY{Meta-variables} ~ 
  & \VAR{T}\SUB{1}, \VAR{T}\SUB{2} <: \NAMEHYPER{../..}{Value-Types}{values} \VAR{T}\SUB{1}\PLUS, \VAR{T}\SUB{2}\PLUS <: \NAMEHYPER{../..}{Value-Types}{values}\PLUS \VAR{T}\STAR, \VAR{T}\SUB{1}\STAR, \VAR{T}\SUB{2}\STAR <: \NAMEHYPER{../..}{Value-Types}{values}\STAR
\end{align*}
\begin{align*}
  \KEY{Datatype} ~ 
  \NAMEDECL{tuples}(\VAR{T}\STAR )  
  ~ ::= ~ & \NAMEDECL{tuple} (\_ : \VAR{T}\STAR)
\end{align*}
$\SHADE{\VAR{T}\STAR}$ can be any sequence of types, including $\SHADE{(  ~  )}$ and $\SHADE{\NAMEHYPER{../..}{Value-Types}{values}\STAR}$.

The values of type $\SHADE{\NAMEREF{tuples}
           ( \VAR{T}\SUB{1},   
             \cdots,   
             \VAR{T}\SUB{n} )}$ are of the form $\SHADE{\NAMEREF{tuple}
           ( \VAR{V}\SUB{1},   
             \cdots,   
             \VAR{V}\SUB{n} )}$
  with $\SHADE{\VAR{V}\SUB{1} : \VAR{T}\SUB{1}}$, \ldots{}, $\SHADE{\VAR{V}\SUB{n} : \VAR{T}\SUB{n}}$.

\begin{align*}
  \KEY{Funcon} ~ 
  & \NAMEDECL{tuple-elements}(\_ : \NAMEREF{tuples}
                                ( \VAR{T}\STAR )) :  \TO ( \VAR{T}\STAR )
\\
  \KEY{Rule} ~ 
    & \NAMEREF{tuple-elements}
        ( \NAMEREF{tuple}
            ( \VAR{V}\STAR : \VAR{T}\STAR ) ) \leadsto
        \VAR{V}\STAR
\end{align*}
\begin{align*}
  \KEY{Funcon} ~ 
  & \NAMEDECL{tuple-zip}(\_ : \NAMEREF{tuples}
                                ( \NAMEHYPER{../..}{Value-Types}{values}\STAR ), \_ : \NAMEREF{tuples}
                                ( \NAMEHYPER{../..}{Value-Types}{values}\STAR )) :  \TO ( \NAMEREF{tuples}
                                                                           ( \NAMEHYPER{../..}{Value-Types}{values},   
                                                                             \NAMEHYPER{../..}{Value-Types}{values} ) )\STAR
\end{align*}
$\SHADE{\NAMEREF{tuple-zip}
           ( \VAR{TV}\SUB{1},   
             \VAR{TV}\SUB{2} )}$ takes two tuples, and returns the sequence of pairs of
  their elements, provided that they have the same length. If they have
  different lengths, the last elements of the longer sequence are ignored.

\begin{align*}
  \KEY{Rule} ~ 
    & \NAMEREF{tuple-zip}
        ( \NAMEREF{tuple}
            ( \VAR{V}\SUB{1} : \VAR{T}\SUB{1},    
              \VAR{V}\SUB{1}\STAR : \VAR{T}\SUB{1}\STAR ),   
          \NAMEREF{tuple}
            ( \VAR{V}\SUB{2} : \VAR{T}\SUB{2},    
              \VAR{V}\SUB{2}\STAR : \VAR{T}\SUB{2}\STAR ) ) \leadsto
        ( \NAMEREF{tuple}
            ( \VAR{V}\SUB{1},   
              \VAR{V}\SUB{2} ),  
          \NAMEREF{tuple-zip}
            ( \NAMEREF{tuple}
                ( \VAR{V}\SUB{1}\STAR ),   
              \NAMEREF{tuple}
                ( \VAR{V}\SUB{2}\STAR ) ) )
\\
  \KEY{Rule} ~ 
    & \NAMEREF{tuple-zip}
        ( \NAMEREF{tuple}
            (  ~  ),   
          \NAMEREF{tuple}
            (  ~  ) ) \leadsto
        (  ~  )
\\
  \KEY{Rule} ~ 
    & \NAMEREF{tuple-zip}
        ( \NAMEREF{tuple}
            ( \VAR{V}\SUB{1}\PLUS : \VAR{T}\SUB{1}\PLUS ),   
          \NAMEREF{tuple}
            (  ~  ) ) \leadsto
        (  ~  )
\\
  \KEY{Rule} ~ 
    & \NAMEREF{tuple-zip}
        ( \NAMEREF{tuple}
            (  ~  ),   
          \NAMEREF{tuple}
            ( \VAR{V}\SUB{2}\PLUS : \VAR{T}\SUB{2}\PLUS ) ) \leadsto
        (  ~  )
\end{align*}
