% 



    OUTLINE
  \tableofcontents
\begin{center}
\rule{3in}{0.4pt}
\end{center}

\subsubsection{Giving}\hypertarget{giving}{}\label{giving}

\begin{align*}
  [ \
  \KEY{Entity} \quad & \NAMEREF{given-value} \\
  \KEY{Funcon} \quad & \NAMEREF{initialise-giving} \\
  \KEY{Funcon} \quad & \NAMEREF{give} \\
  \KEY{Funcon} \quad & \NAMEREF{given} \\
  \KEY{Funcon} \quad & \NAMEREF{no-given} \\
  \KEY{Funcon} \quad & \NAMEREF{left-to-right-map} \\
  \KEY{Funcon} \quad & \NAMEREF{interleave-map} \\
  \KEY{Funcon} \quad & \NAMEREF{left-to-right-repeat} \\
  \KEY{Funcon} \quad & \NAMEREF{interleave-repeat} \\
  \KEY{Funcon} \quad & \NAMEREF{left-to-right-filter} \\
  \KEY{Funcon} \quad & \NAMEREF{interleave-filter} \\
  \KEY{Funcon} \quad & \NAMEREF{fold-left} \\
  \KEY{Funcon} \quad & \NAMEREF{fold-right}
  \ ]
\end{align*}
\begin{align*}
  \KEY{Meta-variables} \quad
  & \VAR{T}, \VAR{T}' <: \NAMEHYPER{../../../Values}{Value-Types}{values} \qquad \\& \VAR{T}\QUERY <: \NAMEHYPER{../../../Values}{Value-Types}{values}\QUERY
\end{align*}
\begin{align*}
  \KEY{Entity} \quad
  & \NAMEDECL{given-value}(\_ : \NAMEHYPER{../../../Values}{Value-Types}{values}\QUERY) \vdash \_ \TRANS  \_
\end{align*}
The given-value entity allows a computation to refer to a single
  previously-computed $\SHADE{\VAR{V} : \NAMEHYPER{../../../Values}{Value-Types}{values}}$. The given value $\SHADE{(   \  )}$ represents 
  the absence of a current given value.

\begin{align*}
  \KEY{Funcon} \quad
  & \NAMEDECL{initialise-giving}(
                       \VAR{X} : (   \  ) \TO \VAR{T}') 
    : (   \  ) \TO \VAR{T}' \\&\quad
    \leadsto \NAMEREF{no-given}
               (  \VAR{X} )
\end{align*}
$\SHADE{\NAMEREF{initialise-giving}
           (  \VAR{X} )}$ ensures that the entities used by the funcons for
  giving are properly initialised.

\begin{align*}
  \KEY{Funcon} \quad
  & \NAMEDECL{give}(
                       \_ : \VAR{T}, \_ : \VAR{T} \TO \VAR{T}') 
    :  \TO \VAR{T}' 
\end{align*}
$\SHADE{\NAMEREF{give}
           (  \VAR{X}, 
                  \VAR{Y} )}$ executes $\SHADE{\VAR{X}}$, possibly referring to the current $\SHADE{\NAMEREF{given}}$ value,
  to compute a value $\SHADE{\VAR{V}}$. It then executes $\SHADE{\VAR{Y}}$ with $\SHADE{\VAR{V}}$ as the $\SHADE{\NAMEREF{given}}$ value,
  to compute the result.

\begin{align*}
  \KEY{Rule} \quad
    & \RULE{
      & \NAMEREF{given-value} (  \VAR{V} ) \vdash \VAR{Y} \TRANS 
          \VAR{Y}'
      }{
      & \NAMEREF{given-value} (  \_\QUERY ) \vdash \NAMEREF{give}
                      (  \VAR{V} : \VAR{T}, 
                             \VAR{Y} ) \TRANS 
          \NAMEREF{give}
            (  \VAR{V}, 
                   \VAR{Y}' )
      }
\\
  \KEY{Rule} \quad
    & \NAMEREF{give}
        (  \_ : \VAR{T}, 
               \VAR{W} : \VAR{T}' ) \leadsto 
        \VAR{W}
\end{align*}
\begin{align*}
  \KEY{Funcon} \quad
  & \NAMEDECL{given} 
    : \VAR{T} \TO \VAR{T} 
\end{align*}
$\SHADE{\NAMEREF{given}}$ refers to the current given value.

\begin{align*}
  \KEY{Rule} \quad
    & \NAMEREF{given-value} (  \VAR{V} : \NAMEHYPER{../../../Values}{Value-Types}{values} ) \vdash \NAMEREF{given} \TRANS 
        \VAR{V}
\\
  \KEY{Rule} \quad
    & \NAMEREF{given-value} (   \  ) \vdash \NAMEREF{given} \TRANS 
        \NAMEHYPER{../../Abnormal}{Failing}{fail}
\end{align*}
\begin{align*}
  \KEY{Funcon} \quad
  & \NAMEDECL{no-given}(
                       \_ : (   \  ) \TO \VAR{T}') 
    : (   \  ) \TO \VAR{T}' 
\end{align*}
$\SHADE{\NAMEREF{no-given}
           (  \VAR{X} )}$ computes $\SHADE{\VAR{X}}$ without references to the current given value.

\begin{align*}
  \KEY{Rule} \quad
    & \RULE{
      & \NAMEREF{given-value} (   \  ) \vdash \VAR{X} \TRANS 
          \VAR{X}'
      }{
      & \NAMEREF{given-value} (  \_\QUERY ) \vdash \NAMEREF{no-given}
                      (  \VAR{X} ) \TRANS 
          \NAMEREF{no-given}
            (  \VAR{X}' )
      }
\\
  \KEY{Rule} \quad
    & \NAMEREF{no-given}
        (  \VAR{U} : \VAR{T}' ) \leadsto 
        \VAR{U}
\end{align*}
\paragraph{Mapping}\hypertarget{mapping}{}\label{mapping}

Maps on collection values can be expressed directly, e.g.,
  $\SHADE{\NAMEHYPER{../../../Values/Composite}{Lists}{list}
           (  \NAMEREF{left-to-right-map}
                   (  \VAR{F}, 
                          \NAMEHYPER{../../../Values/Composite}{Lists}{list-elements}
                           (  \VAR{L} ) ) )}$.

\begin{align*}
  \KEY{Funcon} \quad
  & \NAMEDECL{left-to-right-map}(
                       \_ : \VAR{T} \TO \VAR{T}', \_ : (  \VAR{T} )\STAR) 
    :  \TO (  \VAR{T}' )\STAR 
\end{align*}
$\SHADE{\NAMEREF{left-to-right-map}
           (  \VAR{F}, 
                  \VAR{V}\STAR )}$ computes $\SHADE{\VAR{F}}$ for each value in $\SHADE{\VAR{V}\STAR}$ from left
  to right, returning the sequence of resulting values.

\begin{align*}
  \KEY{Rule} \quad
    & \NAMEREF{left-to-right-map}
        (  \VAR{F}, 
               \VAR{V} : \VAR{T}, 
               \VAR{V}\STAR : (  \VAR{T} )\STAR ) \leadsto \\&\quad
        \NAMEHYPER{../.}{Flowing}{left-to-right}
          (  \NAMEREF{give}
                  (  \VAR{V}, 
                         \VAR{F} ), 
                 \NAMEREF{left-to-right-map}
                  (  \VAR{F}, 
                         \VAR{V}\STAR ) )
\\
  \KEY{Rule} \quad
    & \NAMEREF{left-to-right-map}
        (  \_, 
               (   \  ) ) \leadsto 
        (   \  )
\end{align*}
\begin{align*}
  \KEY{Funcon} \quad
  & \NAMEDECL{interleave-map}(
                       \_ : \VAR{T} \TO \VAR{T}', \_ : (  \VAR{T} )\STAR) 
    :  \TO (  \VAR{T}' )\STAR 
\end{align*}
$\SHADE{\NAMEREF{interleave-map}
           (  \VAR{F}, 
                  \VAR{V}\STAR )}$ computes $\SHADE{\VAR{F}}$ for each value in $\SHADE{\VAR{V}\STAR}$ interleaved, 
  returning the sequence of resulting values.

\begin{align*}
  \KEY{Rule} \quad
    & \NAMEREF{interleave-map}
        (  \VAR{F}, 
               \VAR{V} : \VAR{T}, 
               \VAR{V}\STAR : (  \VAR{T} )\STAR ) \leadsto \\&\quad
        \NAMEHYPER{../.}{Flowing}{interleave}
          (  \NAMEREF{give}
                  (  \VAR{V}, 
                         \VAR{F} ), 
                 \NAMEREF{interleave-map}
                  (  \VAR{F}, 
                         \VAR{V}\STAR ) )
\\
  \KEY{Rule} \quad
    & \NAMEREF{interleave-map}
        (  \_, 
               (   \  ) ) \leadsto 
        (   \  )
\end{align*}
\begin{align*}
  \KEY{Funcon} \quad
  & \NAMEDECL{left-to-right-repeat}(
                       \_ : \NAMEHYPER{../../../Values/Primitive}{Integers}{integers} \TO \VAR{T}', \_ : \NAMEHYPER{../../../Values/Primitive}{Integers}{integers}, \_ : \NAMEHYPER{../../../Values/Primitive}{Integers}{integers}) 
    :  \TO (  \VAR{T}' )\STAR 
\end{align*}
$\SHADE{\NAMEREF{left-to-right-repeat}
           (  \VAR{F}, 
                  \VAR{M}, 
                  \VAR{N} )}$ computes $\SHADE{\VAR{F}}$ for each value from $\SHADE{\VAR{M}}$ to $\SHADE{\VAR{N}}$ 
  sequentially, returning the sequence of resulting values.

\begin{align*}
  \KEY{Rule} \quad
    & \RULE{
      & \NAMEHYPER{../../../Values/Primitive}{Integers}{is-less-or-equal}
          (  \VAR{M}, 
                 \VAR{N} ) 
        == \NAMEHYPER{../../../Values/Primitive}{Booleans}{true}
      }{
      & \NAMEREF{left-to-right-repeat}
          (  \VAR{F}, 
                 \VAR{M} : \NAMEHYPER{../../../Values/Primitive}{Integers}{integers}, 
                 \VAR{N} : \NAMEHYPER{../../../Values/Primitive}{Integers}{integers} ) \leadsto \\&\quad
          \NAMEHYPER{../.}{Flowing}{left-to-right}
            (  \NAMEREF{give}
                    (  \VAR{M}, 
                           \VAR{F} ), 
                   \NAMEREF{left-to-right-repeat}
                    (  \VAR{F}, 
                           \NAMEHYPER{../../../Values/Primitive}{Integers}{int-add}
                            (  \VAR{M}, 
                                   1 ), 
                           \VAR{N} ) )
      }
\\
  \KEY{Rule} \quad
    & \RULE{
      & \NAMEHYPER{../../../Values/Primitive}{Integers}{is-less-or-equal}
          (  \VAR{M}, 
                 \VAR{N} ) 
        == \NAMEHYPER{../../../Values/Primitive}{Booleans}{false}
      }{
      & \NAMEREF{left-to-right-repeat}
          (  \_, 
                 \VAR{M} : \NAMEHYPER{../../../Values/Primitive}{Integers}{integers}, 
                 \VAR{N} : \NAMEHYPER{../../../Values/Primitive}{Integers}{integers} ) \leadsto 
          (   \  )
      }
\end{align*}
\begin{align*}
  \KEY{Funcon} \quad
  & \NAMEDECL{interleave-repeat}(
                       \_ : \NAMEHYPER{../../../Values/Primitive}{Integers}{integers} \TO \VAR{T}', \_ : \NAMEHYPER{../../../Values/Primitive}{Integers}{integers}, \_ : \NAMEHYPER{../../../Values/Primitive}{Integers}{integers}) 
    :  \TO (  \VAR{T}' )\STAR 
\end{align*}
$\SHADE{\NAMEREF{interleave-repeat}
           (  \VAR{F}, 
                  \VAR{M}, 
                  \VAR{N} )}$ computes $\SHADE{\VAR{F}}$ for each value from $\SHADE{\VAR{M}}$ to $\SHADE{\VAR{N}}$ 
  interleaved, returning the sequence of resulting values.

\begin{align*}
  \KEY{Rule} \quad
    & \RULE{
      & \NAMEHYPER{../../../Values/Primitive}{Integers}{is-less-or-equal}
          (  \VAR{M}, 
                 \VAR{N} ) 
        == \NAMEHYPER{../../../Values/Primitive}{Booleans}{true}
      }{
      & \NAMEREF{interleave-repeat}
          (  \VAR{F}, 
                 \VAR{M} : \NAMEHYPER{../../../Values/Primitive}{Integers}{integers}, 
                 \VAR{N} : \NAMEHYPER{../../../Values/Primitive}{Integers}{integers} ) \leadsto \\&\quad
          \NAMEHYPER{../.}{Flowing}{interleave}
            (  \NAMEREF{give}
                    (  \VAR{M}, 
                           \VAR{F} ), 
                   \NAMEREF{interleave-repeat}
                    (  \VAR{F}, 
                           \NAMEHYPER{../../../Values/Primitive}{Integers}{int-add}
                            (  \VAR{M}, 
                                   1 ), 
                           \VAR{N} ) )
      }
\\
  \KEY{Rule} \quad
    & \RULE{
      & \NAMEHYPER{../../../Values/Primitive}{Integers}{is-less-or-equal}
          (  \VAR{M}, 
                 \VAR{N} ) 
        == \NAMEHYPER{../../../Values/Primitive}{Booleans}{false}
      }{
      & \NAMEREF{interleave-repeat}
          (  \_, 
                 \VAR{M} : \NAMEHYPER{../../../Values/Primitive}{Integers}{integers}, 
                 \VAR{N} : \NAMEHYPER{../../../Values/Primitive}{Integers}{integers} ) \leadsto 
          (   \  )
      }
\end{align*}
\paragraph{Filtering}\hypertarget{filtering}{}\label{filtering}

Filters on collections of values can be expressed directly, e.g., 
 $\SHADE{\NAMEHYPER{../../../Values/Composite}{Lists}{list}
           (  \NAMEREF{left-to-right-filter}
                   (  \VAR{P}, 
                          \NAMEHYPER{../../../Values/Composite}{Lists}{list-elements}
                           (  \VAR{L} ) ) )}$ to filter a list $\SHADE{\VAR{L}}$.

\begin{align*}
  \KEY{Funcon} \quad
  & \NAMEDECL{left-to-right-filter}(
                       \_ : \VAR{T} \TO \NAMEHYPER{../../../Values/Primitive}{Booleans}{booleans}, \_ : (  \VAR{T} )\STAR) 
    :  \TO (  \VAR{T} )\STAR 
\end{align*}
$\SHADE{\NAMEREF{left-to-right-filter}
           (  \VAR{P}, 
                  \VAR{V}\STAR )}$ computes $\SHADE{\VAR{P}}$ for each value in $\SHADE{\VAR{V}\STAR}$ from left
  to right, returning the sequence of argument values for which the result is
  $\SHADE{\NAMEHYPER{../../../Values/Primitive}{Booleans}{true}}$.

\begin{align*}
  \KEY{Rule} \quad
    & \NAMEREF{left-to-right-filter}
        (  \VAR{P}, 
               \VAR{V} : \VAR{T}, 
               \VAR{V}\STAR : (  \VAR{T} )\STAR ) \leadsto \\&\quad
        \NAMEHYPER{../.}{Flowing}{left-to-right}
          (  \NAMEHYPER{../../../Values}{Value-Types}{when-true}
                  (  \NAMEREF{give}
                          (  \VAR{V}, 
                                 \VAR{P} ), 
                         \VAR{V} ), 
                 \NAMEREF{left-to-right-filter}
                  (  \VAR{P}, 
                         \VAR{V}\STAR ) )
\\
  \KEY{Rule} \quad
    & \NAMEREF{left-to-right-filter}
        (  \_ ) \leadsto 
        (   \  )
\end{align*}
\begin{align*}
  \KEY{Funcon} \quad
  & \NAMEDECL{interleave-filter}(
                       \_ : \VAR{T} \TO \NAMEHYPER{../../../Values/Primitive}{Booleans}{booleans}, \_ : (  \VAR{T} )\STAR) 
    :  \TO (  \VAR{T} )\STAR 
\end{align*}
$\SHADE{\NAMEREF{interleave-filter}
           (  \VAR{P}, 
                  \VAR{V}\STAR )}$ computes $\SHADE{\VAR{P}}$ for each value in $\SHADE{\VAR{V}\STAR}$ interleaved,
  returning the sequence of argument values for which the result is $\SHADE{\NAMEHYPER{../../../Values/Primitive}{Booleans}{true}}$.

\begin{align*}
  \KEY{Rule} \quad
    & \NAMEREF{interleave-filter}
        (  \VAR{P}, 
               \VAR{V} : \VAR{T}, 
               \VAR{V}\STAR : (  \VAR{T} )\STAR ) \leadsto \\&\quad
        \NAMEHYPER{../.}{Flowing}{interleave}
          (  \NAMEHYPER{../../../Values}{Value-Types}{when-true}
                  (  \NAMEREF{give}
                          (  \VAR{V}, 
                                 \VAR{P} ), 
                         \VAR{V} ), 
                 \NAMEREF{interleave-filter}
                  (  \VAR{P}, 
                         \VAR{V}\STAR ) )
\\
  \KEY{Rule} \quad
    & \NAMEREF{interleave-filter}
        (  \_ ) \leadsto 
        (   \  )
\end{align*}
\paragraph{Folding}\hypertarget{folding}{}\label{folding}

\begin{align*}
  \KEY{Funcon} \quad
  & \NAMEDECL{fold-left}(
                       \_ : \NAMEHYPER{../../../Values/Composite}{Tuples}{tuples}
                                 (  \VAR{T}, 
                                        \VAR{T}' ) \TO \VAR{T}, \_ : \VAR{T}, \_ : (  \VAR{T}' )\STAR) 
    :  \TO \VAR{T} 
\end{align*}
$\SHADE{\NAMEREF{fold-left}
           (  \VAR{F}, 
                  \VAR{A}, 
                  \VAR{V}\STAR )}$ reduces a sequence $\SHADE{\VAR{V}\STAR}$ to a single value by folding it
  from the left, using $\SHADE{\VAR{A}}$ as the initial accumulator value, and iteratively
  updating the accumulator by giving $\SHADE{\VAR{F}}$ the pair of the accumulator value and
  the first of the remaining arguments.

\begin{align*}
  \KEY{Rule} \quad
    & \NAMEREF{fold-left}
        (  \_, 
               \VAR{A} : \VAR{T}, 
               (   \  ) ) \leadsto 
        \VAR{A}
\\
  \KEY{Rule} \quad
    & \NAMEREF{fold-left}
        (  \VAR{F}, 
               \VAR{A} : \VAR{T}, 
               \VAR{V} : \VAR{T}', 
               \VAR{V}\STAR : (  \VAR{T}' )\STAR ) \leadsto 
        \NAMEREF{fold-left}
          (  \VAR{F}, 
                 \NAMEREF{give}
                  (  \NAMEHYPER{../../../Values/Composite}{Tuples}{tuple}
                          (  \VAR{A}, 
                                 \VAR{V} ), 
                         \VAR{F} ), 
                 \VAR{V}\STAR )
\end{align*}
\begin{align*}
  \KEY{Funcon} \quad
  & \NAMEDECL{fold-right}(
                       \_ : \NAMEHYPER{../../../Values/Composite}{Tuples}{tuples}
                                 (  \VAR{T}, 
                                        \VAR{T}' ) \TO \VAR{T}', \_ : \VAR{T}', \_ : (  \VAR{T} )\STAR) 
    :  \TO \VAR{T}' 
\end{align*}
$\SHADE{\NAMEREF{fold-right}
           (  \VAR{F}, 
                  \VAR{A}, 
                  \VAR{V}\STAR )}$ reduces a sequence $\SHADE{\VAR{V}\STAR}$ to a single value by folding it
  from the right, using $\SHADE{\VAR{A}}$ as the initial accumulator value, and iteratively
  updating the accumulator by giving $\SHADE{\VAR{F}}$ the pair of the the last of the 
  remaining arguments and the accumulator value.

\begin{align*}
  \KEY{Rule} \quad
    & \NAMEREF{fold-right}
        (  \_, 
               \VAR{A} : \VAR{T}', 
               (   \  ) ) \leadsto 
        \VAR{A}
\\
  \KEY{Rule} \quad
    & \NAMEREF{fold-right}
        (  \VAR{F}, 
               \VAR{A} : \VAR{T}', 
               \VAR{V}\STAR : (  \VAR{T} )\STAR, 
               \VAR{V} : \VAR{T} ) \leadsto 
        \NAMEREF{give}
          (  \NAMEHYPER{../../../Values/Composite}{Tuples}{tuple}
                  (  \VAR{V}, 
                         \NAMEREF{fold-right}
                          (  \VAR{F}, 
                                 \VAR{A}, 
                                 \VAR{V}\STAR ) ), 
                 \VAR{F} )
\end{align*}
% 


