\subsubsection*{Linking}\hypertarget{linking}{}\label{linking}

\begin{align*}
  [ ~ 
  \KEY{Datatype} ~ & \NAMEREF{links} \\
  \KEY{Funcon} ~ & \NAMEREF{initialise-linking} \\
  \KEY{Funcon} ~ & \NAMEREF{link} \\
  \KEY{Funcon} ~ & \NAMEREF{fresh-link} \\
  \KEY{Funcon} ~ & \NAMEREF{fresh-initialised-link} \\
  \KEY{Alias} ~ & \NAMEREF{fresh-init-link} \\
  \KEY{Funcon} ~ & \NAMEREF{set-link} \\
  \KEY{Funcon} ~ & \NAMEREF{follow-if-link}
  ~ ]
\end{align*}
\begin{align*}
  \KEY{Meta-variables} ~ 
  & \VAR{T} <: \NAMEHYPER{../../../Values}{Value-Types}{values}
\end{align*}
\begin{align*}
  \KEY{Datatype} ~ 
  \NAMEDECL{links}  
  ~ ::= ~ & \NAMEDECL{link} (\_ : \NAMEHYPER{../.}{Storing}{variables})
\end{align*}
\begin{align*}
  \KEY{Funcon} ~ 
  & \NAMEDECL{initialise-linking}(\VAR{X} :  \TO \VAR{T}) :  \TO \VAR{T} \\
  & \quad \leadsto \NAMEHYPER{../.}{Storing}{initialise-storing}
                     ( \VAR{X} )
\end{align*}
$\SHADE{\NAMEREF{initialise-linking}
           ( \VAR{X} )}$ ensures that the entities used by the funcons for
  linking are properly initialised.

\begin{align*}
  \KEY{Funcon} ~ 
  & \NAMEDECL{fresh-link}(\VAR{T} ) :  \TO \NAMEREF{links} \\
  & \quad \leadsto \NAMEREF{link}
                     ( \NAMEHYPER{../.}{Storing}{allocate-variable}
                         ( \VAR{T} ) )
\end{align*}
\begin{align*}
  \KEY{Funcon} ~ 
  & \NAMEDECL{fresh-initialised-link}(\VAR{T} , \VAR{V} : \VAR{T}) :  \TO \NAMEREF{links} \\
  & \quad \leadsto \NAMEREF{link}
                     ( \NAMEHYPER{../.}{Storing}{allocate-initialised-variable}
                         ( \VAR{T}, \\&\quad \quad \quad \quad \quad 
                           \VAR{V} ) )
\\
  \KEY{Alias} ~ 
  & \NAMEDECL{fresh-init-link} = \NAMEREF{fresh-initialised-link}
\end{align*}
\begin{align*}
  \KEY{Funcon} ~ 
  & \NAMEDECL{set-link}(\_ : \NAMEREF{links}, \_ : \VAR{T}) :  \TO \NAMEHYPER{../../../Values/Primitive}{Null}{null-type}
\end{align*}
The value of a link can be set only once.

\begin{align*}
  \KEY{Rule} ~ 
    & \NAMEREF{set-link}
        ( \NAMEREF{link}
            ( \VAR{Var} : \NAMEHYPER{../.}{Storing}{variables} ),   
          \VAR{V} : \VAR{T} ) \leadsto
        \NAMEHYPER{../.}{Storing}{initialise-variable}
          ( \VAR{Var},   
            \VAR{V} )
\end{align*}
\begin{align*}
  \KEY{Funcon} ~ 
  & \NAMEDECL{follow-link}(\_ : \NAMEREF{links}) :  \TO \NAMEHYPER{../../../Values}{Value-Types}{values}
\\
  \KEY{Rule} ~ 
    & \NAMEREF{follow-link}
        ( \NAMEREF{link}
            ( \VAR{Var} : \NAMEHYPER{../.}{Storing}{variables} ) ) \leadsto
        \NAMEHYPER{../.}{Storing}{assigned}
          ( \VAR{Var} )
\end{align*}
\begin{align*}
  \KEY{Funcon} ~ 
  & \NAMEDECL{follow-if-link}(\_ : \NAMEHYPER{../../../Values}{Value-Types}{values}) :  \TO \NAMEHYPER{../../../Values}{Value-Types}{values}
\end{align*}
If $\SHADE{\VAR{V}}$ is a link, $\SHADE{\NAMEREF{follow-if-link}
           ( \VAR{V} )}$ computes the set value, and
  otherwise it evaluates to $\SHADE{\VAR{V}}$.

\begin{align*}
  \KEY{Rule} ~ 
    & \NAMEREF{follow-if-link}
        ( \NAMEREF{link}
            ( \VAR{Var} : \NAMEHYPER{../.}{Storing}{variables} ) ) \leadsto
        \NAMEHYPER{../.}{Storing}{assigned}
          ( \VAR{Var} )
\\
  \KEY{Rule} ~ 
    & \NAMEREF{follow-if-link}
        ( \VAR{V} : \mathop{\sim} \NAMEREF{links} ) \leadsto
        \VAR{V}
\end{align*}
