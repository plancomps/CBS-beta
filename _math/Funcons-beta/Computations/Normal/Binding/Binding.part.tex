% 



    OUTLINE
  \tableofcontents
\begin{center}
\rule{3in}{0.4pt}
\end{center}

\subsubsection{Binding}\hypertarget{binding}{}\label{binding}

\begin{align*}
  [ \
  \KEY{Type} \quad & \NAMEREF{environments} \\
  \KEY{Alias} \quad & \NAMEREF{envs} \\
  \KEY{Datatype} \quad & \NAMEREF{identifiers} \\
  \KEY{Alias} \quad & \NAMEREF{ids} \\
  \KEY{Funcon} \quad & \NAMEREF{identifier-tagged} \\
  \KEY{Alias} \quad & \NAMEREF{id-tagged} \\
  \KEY{Funcon} \quad & \NAMEREF{fresh-identifier} \\
  \KEY{Entity} \quad & \NAMEREF{environment} \\
  \KEY{Alias} \quad & \NAMEREF{env} \\
  \KEY{Funcon} \quad & \NAMEREF{initialise-binding} \\
  \KEY{Funcon} \quad & \NAMEREF{bind-value} \\
  \KEY{Alias} \quad & \NAMEREF{bind} \\
  \KEY{Funcon} \quad & \NAMEREF{unbind} \\
  \KEY{Funcon} \quad & \NAMEREF{bound-directly} \\
  \KEY{Funcon} \quad & \NAMEREF{bound-value} \\
  \KEY{Alias} \quad & \NAMEREF{bound} \\
  \KEY{Funcon} \quad & \NAMEREF{closed} \\
  \KEY{Funcon} \quad & \NAMEREF{scope} \\
  \KEY{Funcon} \quad & \NAMEREF{accumulate} \\
  \KEY{Funcon} \quad & \NAMEREF{collateral} \\
  \KEY{Funcon} \quad & \NAMEREF{bind-recursively} \\
  \KEY{Funcon} \quad & \NAMEREF{recursive}
  \ ]
\end{align*}
\begin{align*}
  \KEY{Meta-variables} \quad
  & \VAR{T} <: \NAMEHYPER{../../../Values}{Value-Types}{values}
\end{align*}
\paragraph{Environments}\hypertarget{environments}{}\label{environments}

\begin{align*}
  \KEY{Type} \quad 
  & \NAMEDECL{environments}  
    \leadsto \NAMEHYPER{../../../Values/Composite}{Maps}{maps}
               (  \NAMEREF{identifiers}, 
                      \NAMEHYPER{../../../Values}{Value-Types}{values}\QUERY )
\\
  \KEY{Alias} \quad
  & \NAMEDECL{envs} = \NAMEREF{environments}
\end{align*}
An environment represents bindings of identifiers to values.
  Mapping an identifier to $\SHADE{(   \  )}$ represents that its binding is hidden.

Circularity in environments (due to recursive bindings) is represented using
  bindings to cut-points called $\SHADE{\NAMEHYPER{../.}{Linking}{links}}$. Funcons are provided for making
  declarations recursive and for referring to bound values without explicit
  mention of links, so their existence can generally be ignored.

\begin{align*}
  \KEY{Datatype} \quad 
  \NAMEDECL{identifiers} 
  \ ::= \ &
  \{ \_ : \NAMEHYPER{../../../Values/Composite}{Strings}{strings} \} \mid \NAMEDECL{identifier-tagged}(
                     \_ : \NAMEREF{identifiers}, \_ : \NAMEHYPER{../../../Values}{Value-Types}{values})
\end{align*}
\begin{align*}
  \KEY{Alias} \quad
  & \NAMEDECL{ids} = \NAMEREF{identifiers}
\\
  \KEY{Alias} \quad
  & \NAMEDECL{id-tagged} = \NAMEREF{identifier-tagged}
\end{align*}
An identifier is either a string of characters, or an identifier tagged with
  some value (e.g., with the identifier of a namespace).

\begin{align*}
  \KEY{Funcon} \quad
  & \NAMEDECL{fresh-identifier} 
    :  \TO \NAMEREF{identifiers} 
\end{align*}
$\SHADE{\NAMEREF{fresh-identifier}}$ computes an identifier distinct from all previously
  computed identifiers.

\begin{align*}
  \KEY{Rule} \quad
    & \NAMEREF{fresh-identifier} \leadsto 
        \NAMEREF{identifier-tagged}
          (  \STRING{generated}, 
                 \NAMEHYPER{../.}{Generating}{fresh-atom} )
\end{align*}
\paragraph{Current bindings}\hypertarget{current-bindings}{}\label{current-bindings}

\begin{align*}
  \KEY{Entity} \quad
  & \NAMEDECL{environment}(\_ : \NAMEREF{environments}) \vdash \_ \TRANS  \_
\end{align*}
\begin{align*}
  \KEY{Alias} \quad
  & \NAMEDECL{env} = \NAMEREF{environment}
\end{align*}
The environment entity allows a computation to refer to the current bindings
  of identifiers to values.

\begin{align*}
  \KEY{Funcon} \quad
  & \NAMEDECL{initialise-binding}(
                       \VAR{X} :  \TO \VAR{T}) 
    :  \TO \VAR{T} \\&\quad
    \leadsto \NAMEHYPER{../.}{Linking}{initialise-linking}
               (  \NAMEHYPER{../.}{Generating}{initialise-generating}
                       (  \NAMEREF{closed}
                               (  \VAR{X} ) ) )
\end{align*}
$\SHADE{\NAMEREF{initialise-binding}
           (  \VAR{X} )}$ ensures that $\SHADE{\VAR{X}}$ does not depend on non-local bindings.
  It also ensures that the linking entity (used to represent potentially cyclic
  bindings) and the generating entity (for creating fresh identifiers) are 
  initialised.

\begin{align*}
  \KEY{Funcon} \quad
  & \NAMEDECL{bind-value}(
                       \VAR{I} : \NAMEREF{identifiers}, \VAR{V} : \NAMEHYPER{../../../Values}{Value-Types}{values}) 
    :  \TO \NAMEREF{environments} \\&\quad
    \leadsto \{ \VAR{I} \mapsto 
                  \VAR{V} \}
\\
  \KEY{Alias} \quad
  & \NAMEDECL{bind} = \NAMEREF{bind-value}
\end{align*}
$\SHADE{\NAMEREF{bind-value}
           (  \VAR{I}, 
                  \VAR{X} )}$ computes the environment that binds only $\SHADE{\VAR{I}}$ to the value
  computed by $\SHADE{\VAR{X}}$.

\begin{align*}
  \KEY{Funcon} \quad
  & \NAMEDECL{unbind}(
                       \VAR{I} : \NAMEREF{identifiers}) 
    :  \TO \NAMEREF{environments} \\&\quad
    \leadsto \{ \VAR{I} \mapsto 
                  (   \  ) \}
\end{align*}
$\SHADE{\NAMEREF{unbind}
           (  \VAR{I} )}$ computes the environment that hides the binding of $\SHADE{\VAR{I}}$.

\begin{align*}
  \KEY{Funcon} \quad
  & \NAMEDECL{bound-directly}(
                       \_ : \NAMEREF{identifiers}) 
    :  \TO \NAMEHYPER{../../../Values}{Value-Types}{values} 
\end{align*}
$\SHADE{\NAMEREF{bound-directly}
           (  \VAR{I} )}$ returns the value to which $\SHADE{\VAR{I}}$ is currently bound, if any,
  and otherwise fails.

$\SHADE{\NAMEREF{bound-directly}
           (  \VAR{I} )}$ does \emph{not} follow links. It is used only in connection with
  recursively-bound values when references are not encapsulated in abstractions.

\begin{align*}
  \KEY{Rule} \quad
    & \RULE{
      & \NAMEHYPER{../../../Values/Composite}{Maps}{lookup}
          (  \VAR{$\rho$}, 
                 \VAR{I} ) \leadsto 
          (  \VAR{V} : \NAMEHYPER{../../../Values}{Value-Types}{values} )
      }{
      & \NAMEREF{environment} (  \VAR{$\rho$} ) \vdash \NAMEREF{bound-directly}
                      (  \VAR{I} : \NAMEREF{identifiers} ) \TRANS 
          \VAR{V}
      }
\\
  \KEY{Rule} \quad
    & \RULE{
      & \NAMEHYPER{../../../Values/Composite}{Maps}{lookup}
          (  \VAR{$\rho$}, 
                 \VAR{I} ) \leadsto 
          (   \  )
      }{
      & \NAMEREF{environment} (  \VAR{$\rho$} ) \vdash \NAMEREF{bound-directly}
                      (  \VAR{I} : \NAMEREF{identifiers} ) \TRANS 
          \NAMEHYPER{../../Abnormal}{Failing}{fail}
      }
\end{align*}
\begin{align*}
  \KEY{Funcon} \quad
  & \NAMEDECL{bound-value}(
                       \VAR{I} : \NAMEREF{identifiers}) 
    :  \TO \NAMEHYPER{../../../Values}{Value-Types}{values} \\&\quad
    \leadsto \NAMEHYPER{../.}{Linking}{follow-if-link}
               (  \NAMEREF{bound-directly}
                       (  \VAR{I} ) )
\\
  \KEY{Alias} \quad
  & \NAMEDECL{bound} = \NAMEREF{bound-value}
\end{align*}
$\SHADE{\NAMEREF{bound-value}
           (  \VAR{I} )}$ inspects the value to which $\SHADE{\VAR{I}}$ is currently bound, if any,
   and otherwise fails. If the value is a link, $\SHADE{\NAMEREF{bound-value}
           (  \VAR{I} )}$ returns the
   value obtained by following the link, if any, and otherwise fails. If the 
   inspected value is not a link, $\SHADE{\NAMEREF{bound-value}
           (  \VAR{I} )}$ returns it.

$\SHADE{\NAMEREF{bound-value}
           (  \VAR{I} )}$ is used for references to non-recursive bindings and to
   recursively-bound values when references are encapsulated in abstractions.

\paragraph{Scope}\hypertarget{scope}{}\label{scope}

\begin{align*}
  \KEY{Funcon} \quad
  & \NAMEDECL{closed}(
                       \VAR{X} :  \TO \VAR{T}) 
    :  \TO \VAR{T} 
\end{align*}
$\SHADE{\NAMEREF{closed}
           (  \VAR{X} )}$ ensures that $\SHADE{\VAR{X}}$ does not depend on non-local bindings.

\begin{align*}
  \KEY{Rule} \quad
    & \RULE{
      & \NAMEREF{environment} (  \NAMEHYPER{../../../Values/Composite}{Maps}{map}
                                     (   \  ) ) \vdash \VAR{X} \TRANS 
          \VAR{X}'
      }{
      & \NAMEREF{environment} (  \_ ) \vdash \NAMEREF{closed}
                      (  \VAR{X} ) \TRANS 
          \NAMEREF{closed}
            (  \VAR{X}' )
      }
\\
  \KEY{Rule} \quad
    & \NAMEREF{closed}
        (  \VAR{V} : \VAR{T} ) \leadsto 
        \VAR{V}
\end{align*}
\begin{align*}
  \KEY{Funcon} \quad
  & \NAMEDECL{scope}(
                       \_ : \NAMEREF{environments}, \_ :  \TO \VAR{T}) 
    :  \TO \VAR{T} 
\end{align*}
$\SHADE{\NAMEREF{scope}
           (  \VAR{D}, 
                  \VAR{X} )}$ executes $\SHADE{\VAR{D}}$ with the current bindings, to compute an environment
  $\SHADE{\VAR{$\rho$}}$ representing local bindings. It then executes $\SHADE{\VAR{X}}$ to compute the result,
  with the current bindings extended by $\SHADE{\VAR{$\rho$}}$, which may shadow or hide previous
  bindings.

$\SHADE{\NAMEREF{closed}
           (  \NAMEREF{scope}
                   (  \VAR{$\rho$}, 
                          \VAR{X} ) )}$ ensures that $\SHADE{\VAR{X}}$ can reference only the bindings
  provided by $\SHADE{\VAR{$\rho$}}$.

\begin{align*}
  \KEY{Rule} \quad
    & \RULE{
      & \NAMEREF{environment} (  \NAMEHYPER{../../../Values/Composite}{Maps}{map-override}
                                     (  \VAR{$\rho$}\SUB{1}, 
                                            \VAR{$\rho$}\SUB{0} ) ) \vdash \VAR{X} \TRANS 
          \VAR{X}'
      }{
      & \NAMEREF{environment} (  \VAR{$\rho$}\SUB{0} ) \vdash \NAMEREF{scope}
                      (  \VAR{$\rho$}\SUB{1} : \NAMEREF{environments}, 
                             \VAR{X} ) \TRANS 
          \NAMEREF{scope}
            (  \VAR{$\rho$}\SUB{1}, 
                   \VAR{X}' )
      }
\\
  \KEY{Rule} \quad
    & \NAMEREF{scope}
        (  \_ : \NAMEREF{environments}, 
               \VAR{V} : \VAR{T} ) \leadsto 
        \VAR{V}
\end{align*}
\begin{align*}
  \KEY{Funcon} \quad
  & \NAMEDECL{accumulate}(
                       \_ : (   \TO \NAMEREF{environments} )\STAR) 
    :  \TO \NAMEREF{environments} 
\end{align*}
$\SHADE{\NAMEREF{accumulate}
           (  \VAR{D}\SUB{1}, 
                  \VAR{D}\SUB{2} )}$ executes $\SHADE{\VAR{D}\SUB{1}}$ with the current bindings, to compute an
  environment $\SHADE{\VAR{$\rho$}\SUB{1}}$ representing some local bindings. It then executes $\SHADE{\VAR{D}\SUB{2}}$ to
  compute an environment $\SHADE{\VAR{$\rho$}\SUB{2}}$ representing further local bindings, with the
  current bindings extended by $\SHADE{\VAR{$\rho$}\SUB{1}}$, which may shadow or hide previous
  current bindings. The result is $\SHADE{\VAR{$\rho$}\SUB{1}}$ extended by $\SHADE{\VAR{$\rho$}\SUB{2}}$, which may shadow
  or hide the bindings of $\SHADE{\VAR{$\rho$}\SUB{1}}$.

$\SHADE{\NAMEREF{accumulate}
           (  \_, 
                  \_ )}$ is associative, with $\SHADE{\NAMEHYPER{../../../Values/Composite}{Maps}{map}
           (   \  )}$ as unit, and extends to any
  number of arguments.

\begin{align*}
  \KEY{Rule} \quad
    & \RULE{
      &  \VAR{D}\SUB{1} \TRANS 
          \VAR{D}\SUB{1}'
      }{
      &  \NAMEREF{accumulate}
                      (  \VAR{D}\SUB{1}, 
                             \VAR{D}\SUB{2} ) \TRANS 
          \NAMEREF{accumulate}
            (  \VAR{D}\SUB{1}', 
                   \VAR{D}\SUB{2} )
      }
\\
  \KEY{Rule} \quad
    & \NAMEREF{accumulate}
        (  \VAR{$\rho$}\SUB{1} : \NAMEREF{environments}, 
               \VAR{D}\SUB{2} ) \leadsto 
        \NAMEREF{scope}
          (  \VAR{$\rho$}\SUB{1}, 
                 \NAMEHYPER{../../../Values/Composite}{Maps}{map-override}
                  (  \VAR{D}\SUB{2}, 
                         \VAR{$\rho$}\SUB{1} ) )
\\
  \KEY{Rule} \quad
    & \NAMEREF{accumulate}
        (   \  ) \leadsto 
        \NAMEHYPER{../../../Values/Composite}{Maps}{map}
          (   \  )
\\
  \KEY{Rule} \quad
    & \NAMEREF{accumulate}
        (  \VAR{D}\SUB{1} ) \leadsto 
        \VAR{D}\SUB{1}
\\
  \KEY{Rule} \quad
    & \NAMEREF{accumulate}
        (  \VAR{D}\SUB{1}, 
               \VAR{D}\SUB{2}, 
               \VAR{D}\PLUS ) \leadsto 
        \NAMEREF{accumulate}
          (  \VAR{D}\SUB{1}, 
                 \NAMEREF{accumulate}
                  (  \VAR{D}\SUB{2}, 
                         \VAR{D}\PLUS ) )
\end{align*}
\begin{align*}
  \KEY{Funcon} \quad
  & \NAMEDECL{collateral}(
                       \VAR{$\rho$}\STAR : \NAMEREF{environments}\STAR) 
    :  \TO \NAMEREF{environments} \\&\quad
    \leadsto \NAMEHYPER{../../Abnormal}{Failing}{checked} \ 
               \NAMEHYPER{../../../Values/Composite}{Maps}{map-unite}
                 (  \VAR{$\rho$}\STAR )
\end{align*}
$\SHADE{\NAMEREF{collateral}
           (  \VAR{D}\SUB{1}, 
                  \cdots )}$ pre-evaluates its arguments with the current bindings,
  and unites the resulting maps, which fails if the domains are not pairwise
  disjoint.

$\SHADE{\NAMEREF{collateral}
           (  \VAR{D}\SUB{1}, 
                  \VAR{D}\SUB{2} )}$ is associative and commutative with $\SHADE{\NAMEHYPER{../../../Values/Composite}{Maps}{map}
           (   \  )}$ as unit, 
  and extends to any number of arguments.

\paragraph{Recurse}\hypertarget{recurse}{}\label{recurse}

\begin{align*}
  \KEY{Funcon} \quad
  & \NAMEDECL{bind-recursively}(
                       \VAR{I} : \NAMEREF{identifiers}, \VAR{E} :  \TO \NAMEHYPER{../../../Values}{Value-Types}{values}) 
    :  \TO \NAMEREF{environments} \\&\quad
    \leadsto \NAMEREF{recursive}
               (  \{  \VAR{I} \}, 
                      \NAMEREF{bind-value}
                       (  \VAR{I}, 
                              \VAR{E} ) )
\end{align*}
$\SHADE{\NAMEREF{bind-recursively}
           (  \VAR{I}, 
                  \VAR{E} )}$ binds $\SHADE{\VAR{I}}$ to a link that refers to the value of $\SHADE{\VAR{E}}$, 
  representing a recursive binding of $\SHADE{\VAR{I}}$ to the value of $\SHADE{\VAR{E}}$.
  Since $\SHADE{\NAMEREF{bound-value}
           (  \VAR{I} )}$ follows links, it should not be executed during the
  evaluation of $\SHADE{\VAR{E}}$.

\begin{align*}
  \KEY{Funcon} \quad
  & \NAMEDECL{recursive}(
                       \VAR{SI} : \NAMEHYPER{../../../Values/Composite}{Sets}{sets}
                                 (  \NAMEREF{identifiers} ), \VAR{D} :  \TO \NAMEREF{environments}) 
    :  \TO \NAMEREF{environments} \\&\quad
    \leadsto \NAMEREF{re-close}
               (  \NAMEREF{bind-to-forward-links}
                       (  \VAR{SI} ), 
                      \VAR{D} )
\end{align*}
$\SHADE{\NAMEREF{recursive}
           (  \VAR{SI}, 
                  \VAR{D} )}$ executes $\SHADE{\VAR{D}}$ with potential recursion on the bindings of 
  the identifiers in the set $\SHADE{\VAR{SI}}$ (which need not be the same as the set of
  identifiers bound by $\SHADE{\VAR{D}}$).

\begin{align*}
  \KEY{Auxiliary Funcon} \quad
  & \NAMEDECL{re-close}(
                       \VAR{M} : \NAMEHYPER{../../../Values/Composite}{Maps}{maps}
                                 (  \NAMEREF{identifiers}, 
                                        \NAMEHYPER{../.}{Linking}{links} ), \VAR{D} :  \TO \NAMEREF{environments}) 
    :  \TO \NAMEREF{environments} \\&\quad
    \leadsto \NAMEREF{accumulate}
               (  \NAMEREF{scope}
                       (  \VAR{M}, 
                              \VAR{D} ), 
                      \NAMEHYPER{../.}{Flowing}{sequential}
                       (  \NAMEREF{set-forward-links}
                               (  \VAR{M} ), 
                              \NAMEHYPER{../../../Values/Composite}{Maps}{map}
                               (   \  ) ) )
\end{align*}
$\SHADE{\NAMEREF{re-close}
           (  \VAR{M}, 
                  \VAR{D} )}$ first executes $\SHADE{\VAR{D}}$ in the scope $\SHADE{\VAR{M}}$, which maps identifiers
  to freshly allocated links. This computes an environment $\SHADE{\VAR{$\rho$}}$ where the bound
  values may contain links, or implicit references to links in abstraction
  values. It then sets the link for each identifier in the domain of $\SHADE{\VAR{M}}$ to
  refer to its bound value in $\SHADE{\VAR{$\rho$}}$, and returns $\SHADE{\VAR{$\rho$}}$ as the result.

\begin{align*}
  \KEY{Auxiliary Funcon} \quad
  & \NAMEDECL{bind-to-forward-links}(
                       \VAR{SI} : \NAMEHYPER{../../../Values/Composite}{Sets}{sets}
                                 (  \NAMEREF{identifiers} )) 
    :  \TO \NAMEHYPER{../../../Values/Composite}{Maps}{maps}
                     (  \NAMEREF{identifiers}, 
                            \NAMEHYPER{../.}{Linking}{links} ) \\&\quad
    \leadsto \NAMEHYPER{../../../Values/Composite}{Maps}{map-unite}
               ( \\&\quad\quad\quad\quad \NAMEHYPER{../.}{Giving}{interleave-map}
                       ( \\&\quad\quad\quad\quad\quad \NAMEREF{bind-value}
                               (  \NAMEHYPER{../.}{Giving}{given}, 
                                      \NAMEHYPER{../.}{Linking}{fresh-link}
                                       (  \NAMEHYPER{../../../Values}{Value-Types}{values} ) ), \\&\quad\quad\quad\quad\quad
                              \NAMEHYPER{../../../Values/Composite}{Sets}{set-elements}
                               (  \VAR{SI} ) ) )
\end{align*}
$\SHADE{\NAMEREF{bind-to-forward-links}
           (  \VAR{SI} )}$ binds each identifier in the set $\SHADE{\VAR{SI}}$ to a
  freshly allocated link.

\begin{align*}
  \KEY{Auxiliary Funcon} \quad
  & \NAMEDECL{set-forward-links}(
                       \VAR{M} : \NAMEHYPER{../../../Values/Composite}{Maps}{maps}
                                 (  \NAMEREF{identifiers}, 
                                        \NAMEHYPER{../.}{Linking}{links} )) 
    :  \TO \NAMEHYPER{../../../Values/Primitive}{Null}{null-type} \\&\quad
    \leadsto \NAMEHYPER{../.}{Flowing}{effect}
               ( \\&\quad\quad\quad\quad \NAMEHYPER{../.}{Giving}{interleave-map}
                       ( \\&\quad\quad\quad\quad\quad \NAMEHYPER{../.}{Linking}{set-link}
                               (  \NAMEHYPER{../../../Values/Composite}{Maps}{map-lookup}
                                       (  \VAR{M}, 
                                              \NAMEHYPER{../.}{Giving}{given} ), 
                                      \NAMEREF{bound-value}
                                       (  \NAMEHYPER{../.}{Giving}{given} ) ), \\&\quad\quad\quad\quad\quad
                              \NAMEHYPER{../../../Values/Composite}{Sets}{set-elements}
                               (  \NAMEHYPER{../../../Values/Composite}{Maps}{map-domain}
                                       (  \VAR{M} ) ) ) )
\end{align*}
For each identifier $\SHADE{\VAR{I}}$ in the domain of $\SHADE{\VAR{M}}$, $\SHADE{\NAMEREF{set-forward-links}
           (  \VAR{M} )}$ sets the 
  link to which $\SHADE{\VAR{I}}$ is mapped by $\SHADE{\VAR{M}}$ to the current bound value of $\SHADE{\VAR{I}}$.

% 


