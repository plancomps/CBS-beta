\subsubsection*{Controlling}\hypertarget{controlling}{}\label{controlling}

\begin{align*}
  [ ~ 
  \KEY{Datatype} ~ & \NAMEREF{continuations} \\
  \KEY{Funcon} ~ & \NAMEREF{continuation} \\
  \KEY{Entity} ~ & \NAMEREF{plug-signal} \\
  \KEY{Funcon} ~ & \NAMEREF{hole} \\
  \KEY{Funcon} ~ & \NAMEREF{resume-continuation} \\
  \KEY{Entity} ~ & \NAMEREF{control-signal} \\
  \KEY{Funcon} ~ & \NAMEREF{control} \\
  \KEY{Funcon} ~ & \NAMEREF{delimit-current-continuation} \\
  \KEY{Alias} ~ & \NAMEREF{delimit-cc}
  ~ ]
\end{align*}
\begin{align*}
  \KEY{Meta-variables} ~ 
  & \VAR{T}, \VAR{T}\SUB{1}, \VAR{T}\SUB{2} <: \NAMEHYPER{../../../Values}{Value-Types}{values}
\end{align*}
\begin{align*}
  \KEY{Datatype} ~ 
  \NAMEDECL{continuations}(\VAR{T}\SUB{1} , \VAR{T}\SUB{2} )  
  ~ ::= ~ & \NAMEDECL{continuation} (\_ : \NAMEHYPER{../../../Values/Abstraction}{Generic}{abstractions}
                                         ( (  ~  ) \TO \VAR{T}\SUB{2} ))
\end{align*}
$\SHADE{\NAMEREF{continuations}
           ( \VAR{T}\SUB{1},   
             \VAR{T}\SUB{2} )}$ consists of abstractions whose bodies contain a $\SHADE{\NAMEREF{hole}}$,
  and which will normally compute a value of type $\SHADE{\VAR{T}\SUB{2}}$ when the $\SHADE{\NAMEREF{hole}}$ is plugged
  with a value of type $\SHADE{\VAR{T}\SUB{1}}$.

\begin{align*}
  \KEY{Entity} ~ 
  & \_ \xrightarrow{\NAMEDECL{plug-signal}(\VAR{V}\QUERY : \NAMEHYPER{../../../Values}{Value-Types}{values}\QUERY)} \_
\end{align*}
A plug-signal contains the value to be filled into a $\SHADE{\NAMEREF{hole}}$ in a continuation,
   thereby allowing a continuation to resume.

\begin{align*}
  \KEY{Funcon} ~ 
  & \NAMEDECL{hole} :  \TO \NAMEHYPER{../../../Values}{Value-Types}{values}
\end{align*}
A $\SHADE{\NAMEREF{hole}}$ in a term cannot proceed until it receives a plug-signal
  containing a value to plug the hole.

\begin{align*}
  \KEY{Rule} ~ 
    &  \NAMEREF{hole} \xrightarrow{\NAMEREF{plug-signal}( \VAR{V} )}_{} 
        \VAR{V}
\end{align*}
\begin{align*}
  \KEY{Funcon} ~ 
  & \NAMEDECL{resume-continuation}(\VAR{K} : \NAMEREF{continuations}
                                ( \VAR{T}\SUB{1},   
                                  \VAR{T}\SUB{2} ), \VAR{V} : \VAR{T}\SUB{1}) :  \TO \VAR{T}\SUB{2}
\end{align*}
$\SHADE{\NAMEREF{resume-continuation}
           ( \VAR{K},   
             \VAR{V} )}$ resumes a continuation $\SHADE{\VAR{K}}$ by plugging the value
 $\SHADE{\VAR{V}}$ into the $\SHADE{\NAMEREF{hole}}$ in the continuation.

\begin{align*}
  \KEY{Rule} ~ 
    & \RULE{
       \VAR{X} \xrightarrow{\NAMEREF{plug-signal}( \VAR{V} )}_{} 
        \VAR{X}'
      }{
      &  \NAMEREF{resume-continuation}
                      ( \NAMEREF{continuation}
                          ( \NAMEHYPER{../../../Values/Abstraction}{Generic}{abstraction}
                              ( \VAR{X} ) ),   
                        \VAR{V} : \VAR{T} ) \xrightarrow{\NAMEREF{plug-signal}(  ~  )}_{} 
          \VAR{X}'
      }
\end{align*}
\begin{align*}
  \KEY{Entity} ~ 
  & \_ \xrightarrow{\NAMEDECL{control-signal}(\VAR{F}\QUERY : ( \NAMEHYPER{../../../Values/Abstraction}{Functions}{functions}
                                                                  ( \NAMEREF{continuations}
                                                                      ( \VAR{T}\SUB{1},    
                                                                        \VAR{T}\SUB{2} ),   
                                                                    \VAR{T}\SUB{2} ) )\QUERY)} \_
\end{align*}
A control-signal contains the function to which control is about to be passed
   by the enclosing $\SHADE{\NAMEREF{delimit-current-continuation}
           ( \VAR{X} )}$.

\begin{align*}
  \KEY{Funcon} ~ 
  & \NAMEDECL{control}(\VAR{F} : \NAMEHYPER{../../../Values/Abstraction}{Functions}{functions}
                                ( \NAMEREF{continuations}
                                    ( \VAR{T}\SUB{1},    
                                      \VAR{T}\SUB{2} ),   
                                  \VAR{T}\SUB{2} )) :  \TO \VAR{T}\SUB{1}
\end{align*}
$\SHADE{\NAMEREF{control}
           ( \VAR{F} )}$ emits a control-signal that, when handled by an enclosing
  $\SHADE{\NAMEREF{delimit-current-continuation}
           ( \VAR{X} )}$, will apply $\SHADE{\VAR{F}}$ to the current continuation of
  $\SHADE{\NAMEREF{control}
           ( \VAR{F} )}$, (rather than proceeding with that current continuation).

\begin{align*}
  \KEY{Rule} ~ 
    &  \NAMEREF{control}
                    ( \VAR{F} : \NAMEHYPER{../../../Values/Abstraction}{Functions}{functions}
                                  ( \_,    
                                    \_ ) ) \xrightarrow{\NAMEREF{control-signal}( \VAR{F} )}_{} 
        \NAMEREF{hole}
\end{align*}
\begin{align*}
  \KEY{Funcon} ~ 
  & \NAMEDECL{delimit-current-continuation}(\VAR{X} :  \TO \VAR{T}) :  \TO \VAR{T}
\\
  \KEY{Alias} ~ 
  & \NAMEDECL{delimit-cc} = \NAMEREF{delimit-current-continuation}
\end{align*}
$\SHADE{\NAMEREF{delimit-current-continuation}
           ( \VAR{X} )}$ delimits the scope of captured continuations.

\begin{align*}
  \KEY{Rule} ~ 
    & \NAMEREF{delimit-current-continuation}
        ( \VAR{V} : \VAR{T} ) \leadsto
        \VAR{V}
\\
  \KEY{Rule} ~ 
    & \RULE{
       \VAR{X} \xrightarrow{\NAMEREF{control-signal}(  ~  )}_{} 
        \VAR{X}'
      }{
      &  \NAMEREF{delimit-current-continuation}
                      ( \VAR{X} ) \xrightarrow{\NAMEREF{control-signal}(  ~  )}_{} 
          \NAMEREF{delimit-current-continuation}
            ( \VAR{X}' )
      }
\\
  \KEY{Rule} ~ 
    & \RULE{
       \VAR{X} \xrightarrow{\NAMEREF{control-signal}( \VAR{F} )}_{} 
        \VAR{X}'
      }{
      &  \NAMEREF{delimit-current-continuation}
                      ( \VAR{X} ) \xrightarrow{\NAMEREF{control-signal}(  ~  )}_{} 
          \NAMEREF{delimit-current-continuation}
            ( \NAMEHYPER{../../../Values/Abstraction}{Functions}{apply}
                ( \VAR{F},    
                  \NAMEREF{continuation} ~
                    \NAMEHYPER{../../../Values/Abstraction}{Generic}{closure}
                      ( \VAR{X}' ) ) )
      }
\end{align*}
