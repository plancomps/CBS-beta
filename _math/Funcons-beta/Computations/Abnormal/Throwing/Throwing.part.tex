% 


\begin{center}
\rule{3in}{0.4pt}
\end{center}

\subsubsection{Throwing}\hypertarget{throwing}{}\label{throwing}

\begin{align*}
  [ \
  \KEY{Datatype} \quad & \NAMEREF{throwing} \\
  \KEY{Funcon} \quad & \NAMEREF{thrown} \\
  \KEY{Funcon} \quad & \NAMEREF{finalise-throwing} \\
  \KEY{Funcon} \quad & \NAMEREF{throw} \\
  \KEY{Funcon} \quad & \NAMEREF{handle-thrown} \\
  \KEY{Funcon} \quad & \NAMEREF{handle-recursively} \\
  \KEY{Funcon} \quad & \NAMEREF{catch-else-throw}
  \ ]
\end{align*}
\begin{align*}
  \KEY{Meta-variables} \quad
  & \VAR{R}, \VAR{S}, \VAR{T}, \VAR{T}', \VAR{T}'' <: \NAMEHYPER{../../../Values}{Value-Types}{values}
\end{align*}
\begin{align*}
  \KEY{Datatype} \quad 
  \NAMEDECL{throwing} 
  \ ::= \ & \NAMEDECL{thrown}(
                               \_ : \NAMEHYPER{../../../Values}{Value-Types}{values})
\end{align*}
$\SHADE{\NAMEREF{thrown}
           (  \VAR{V} )}$ is a reason for abrupt termination.

\begin{align*}
  \KEY{Funcon} \quad
  & \NAMEDECL{finalise-throwing}(
                       \VAR{X} :  \TO \VAR{T}) 
    :  \TO \VAR{T}  \mid \NAMEHYPER{../../../Values/Primitive}{Null}{null-type} \\&\quad
    \leadsto \NAMEHYPER{../.}{Abrupting}{finalise-abrupting}
               (  \VAR{X} )
\end{align*}
$\SHADE{\NAMEREF{finalise-throwing}
           (  \VAR{X} )}$ handles abrupt termination of $\SHADE{\VAR{X}}$ due to
  executing $\SHADE{\NAMEREF{throw}
           (  \VAR{V} )}$.

\begin{align*}
  \KEY{Funcon} \quad
  & \NAMEDECL{throw}(
                       \VAR{V} : \VAR{T}) 
    :  \TO \NAMEHYPER{../../../Values}{Value-Types}{empty-type} \\&\quad
    \leadsto \NAMEHYPER{../.}{Abrupting}{abrupt}
               (  \NAMEREF{thrown}
                       (  \VAR{V} ) )
\end{align*}
$\SHADE{\NAMEREF{throw}
           (  \VAR{V} )}$ abruptly terminates all enclosing computations uTil it is handled.

\begin{align*}
  \KEY{Funcon} \quad
  & \NAMEDECL{handle-thrown}(
                       \_ : \VAR{T}' \TO \VAR{T}, \_ : \VAR{T}'' \TO \VAR{T}) 
    : \VAR{T}' \TO \VAR{T} 
\end{align*}
$\SHADE{\NAMEREF{handle-thrown}
           (  \VAR{X}, 
                  \VAR{Y} )}$ first evaluates $\SHADE{\VAR{X}}$. If $\SHADE{\VAR{X}}$ terminates normally with
  value $\SHADE{\VAR{V}}$, then $\SHADE{\VAR{V}}$ is returned and $\SHADE{\VAR{Y}}$ is ignored. If $\SHADE{\VAR{X}}$ terminates abruptly
  with a thrown eTity having value $\SHADE{\VAR{V}}$, then $\SHADE{\VAR{Y}}$ is executed with $\SHADE{\VAR{V}}$ as
  $\SHADE{\NAMEHYPER{../../Normal}{Giving}{given}}$ value.

$\SHADE{\NAMEREF{handle-thrown}
           (  \VAR{X}, 
                  \VAR{Y} )}$ is associative, with $\SHADE{\NAMEREF{throw}
           (  \NAMEHYPER{../../Normal}{Giving}{given} )}$ as unit.
  $\SHADE{\NAMEREF{handle-thrown}
           (  \VAR{X}, 
                  \NAMEHYPER{../.}{Failing}{else}
                   (  \VAR{Y}, 
                          \NAMEREF{throw}
                           (  \NAMEHYPER{../../Normal}{Giving}{given} ) ) )}$ ensures that if $\SHADE{\VAR{Y}}$ fails, the
  thrown value is re-thrown.

\begin{align*}
  \KEY{Rule} \quad
    & \RULE{
      &  \VAR{X} \xrightarrow{\NAMEHYPER{../.}{Abrupting}{abrupted}(   \  )}_{} 
          \VAR{X}'
      }{
      &  \NAMEREF{handle-thrown}
                      (  \VAR{X}, 
                             \VAR{Y} ) \xrightarrow{\NAMEHYPER{../.}{Abrupting}{abrupted}(   \  )}_{} 
          \NAMEREF{handle-thrown}
            (  \VAR{X}', 
                   \VAR{Y} )
      }
\\
  \KEY{Rule} \quad
    & \RULE{
      &  \VAR{X} \xrightarrow{\NAMEHYPER{../.}{Abrupting}{abrupted}(  \NAMEREF{thrown}
                                                                                  (  \VAR{V}'' : \NAMEHYPER{../../../Values}{Value-Types}{values} ) )}_{} 
          \VAR{X}'
      }{
      &  \NAMEREF{handle-thrown}
                      (  \VAR{X}, 
                             \VAR{Y} ) \xrightarrow{\NAMEHYPER{../.}{Abrupting}{abrupted}(   \  )}_{} 
          \NAMEHYPER{../../Normal}{Giving}{give}
            (  \VAR{V}'', 
                   \VAR{Y} )
      }
\\
  \KEY{Rule} \quad
    & \RULE{
      &  \VAR{X} \xrightarrow{\NAMEHYPER{../.}{Abrupting}{abrupted}(  \VAR{V}' : \mathop{\sim} \NAMEREF{throwing} )}_{} 
          \VAR{X}'
      }{
      &  \NAMEREF{handle-thrown}
                      (  \VAR{X}, 
                             \VAR{Y} ) \xrightarrow{\NAMEHYPER{../.}{Abrupting}{abrupted}(  \VAR{V}' )}_{} 
          \NAMEREF{handle-thrown}
            (  \VAR{X}', 
                   \VAR{Y} )
      }
\\
  \KEY{Rule} \quad
    & \NAMEREF{handle-thrown}
        (  \VAR{V} : \VAR{T}, 
               \VAR{Y} ) \leadsto 
        \VAR{V}
\end{align*}
\begin{align*}
  \KEY{Funcon} \quad
  & \NAMEDECL{handle-recursively}(
                       \VAR{X} : \VAR{S} \TO \VAR{T}, \VAR{Y} : \VAR{R} \TO \VAR{T}) 
    : \VAR{S} \TO \VAR{T} \\&\quad
    \leadsto \NAMEREF{handle-thrown}
               (  \VAR{X}, 
                      \NAMEHYPER{../.}{Failing}{else}
                       (  \NAMEREF{handle-recursively}
                               (  \VAR{Y}, 
                                      \VAR{Y} ), 
                              \NAMEREF{throw}
                               (  \NAMEHYPER{../../Normal}{Giving}{given} ) ) )
\end{align*}
$\SHADE{\NAMEREF{handle-recursively}
           (  \VAR{X}, 
                  \VAR{Y} )}$ behaves similarly to $\SHADE{\NAMEREF{handle-thrown}
           (  \VAR{X}, 
                  \VAR{Y} )}$, except
  that another copy of the handler attempts to handle any values thrown by $\SHADE{\VAR{Y}}$.
  Thus, many thrown values may get handled by the same handler.

\begin{align*}
  \KEY{Funcon} \quad
  & \NAMEDECL{catch-else-throw}(
                       \VAR{P} : \NAMEHYPER{../../../Values}{Value-Types}{values}, \VAR{Y} :  \TO \VAR{T}) 
    :  \TO \VAR{T} \\&\quad
    \leadsto \NAMEHYPER{../.}{Failing}{else}
               (  \NAMEHYPER{../../../Values/Abstraction}{Patterns}{case-match}
                       (  \VAR{P}, 
                              \VAR{Y} ), 
                      \NAMEREF{throw}
                       (  \NAMEHYPER{../../Normal}{Giving}{given} ) )
\end{align*}
$\SHADE{\NAMEREF{handle-thrown}
           (  \VAR{X}, 
                  \NAMEREF{catch-else-throw}
                   (  \VAR{P}, 
                          \VAR{Y} ) )}$ handles those values thrown by $\SHADE{\VAR{X}}$
   that match pattern $\SHADE{\VAR{P}}$.  Other thrown values are re-thrown.

% 


