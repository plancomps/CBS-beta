% 


\begin{center}
\rule{3in}{0.4pt}
\end{center}

\subsubsection{Variants}\hypertarget{variants}{}\label{variants}

\begin{align*}
  [ \
  \KEY{Datatype} \quad & \NAMEREF{variants} \\
  \KEY{Funcon} \quad & \NAMEREF{variant} \\
  \KEY{Funcon} \quad & \NAMEREF{variant-id} \\
  \KEY{Funcon} \quad & \NAMEREF{variant-value}
  \ ]
\end{align*}
\begin{align*}
  \KEY{Meta-variables} \quad
  & \VAR{T} <: \NAMEHYPER{../..}{Value-Types}{values}
\end{align*}
\begin{align*}
  \KEY{Datatype} \quad 
  \NAMEDECL{variants}(
                     \VAR{T} ) 
  \ ::= \ & \NAMEDECL{variant}(
                               \_ : \NAMEHYPER{../../../Computations/Normal}{Binding}{identifiers}, \_ : \VAR{T})
\end{align*}
A value of type $\SHADE{\NAMEREF{variants}
           (  \VAR{T} )}$ is a pair formed from an identifier and 
  a value of type $\SHADE{\VAR{T}}$.

\begin{align*}
  \KEY{Funcon} \quad
  & \NAMEDECL{variant-id}(
                       \_ : \NAMEREF{variants}
                                 (  \VAR{T} )) 
    :  \TO \NAMEHYPER{../../../Computations/Normal}{Binding}{identifiers} 
\\
  \KEY{Rule} \quad
    & \NAMEREF{variant-id}
        (  \NAMEREF{variant}
                (  \VAR{I} : \NAMEHYPER{../../../Computations/Normal}{Binding}{identifiers}, 
                       \_ : \VAR{T} ) ) \leadsto 
        \VAR{I}
\end{align*}
\begin{align*}
  \KEY{Funcon} \quad
  & \NAMEDECL{variant-value}(
                       \_ : \NAMEREF{variants}
                                 (  \VAR{T} )) 
    :  \TO \VAR{T} 
\\
  \KEY{Rule} \quad
    & \NAMEREF{variant-value}
        (  \NAMEREF{variant}
                (  \_ : \NAMEHYPER{../../../Computations/Normal}{Binding}{identifiers}, 
                       \VAR{V} : \VAR{T} ) ) \leadsto 
        \VAR{V}
\end{align*}
% 


