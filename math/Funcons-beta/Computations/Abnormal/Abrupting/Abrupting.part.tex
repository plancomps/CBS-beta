% 


\begin{center}
\rule{3in}{0.4pt}
\end{center}

\subsubsection{Abruptly terminating}\hypertarget{abruptly-terminating}{}\label{abruptly-terminating}

\begin{align*}
  [ \
  \KEY{Funcon} \quad & \NAMEREF{stuck} \\
  \KEY{Entity} \quad & \NAMEREF{abrupted} \\
  \KEY{Funcon} \quad & \NAMEREF{finalise-abrupting} \\
  \KEY{Funcon} \quad & \NAMEREF{abrupt} \\
  \KEY{Funcon} \quad & \NAMEREF{handle-abrupt} \\
  \KEY{Funcon} \quad & \NAMEREF{finally}
  \ ]
\end{align*}
\begin{align*}
  \KEY{Meta-variables} \quad
  & \VAR{T}, \VAR{T}', \VAR{T}'' <: \NAMEHYPER{../../../Values}{Value-Types}{values}
\end{align*}
\begin{align*}
  \KEY{Funcon} \quad
  & \NAMEDECL{stuck} 
    :  \TO \NAMEHYPER{../../../Values}{Value-Types}{empty-type} 
\end{align*}
$\SHADE{\NAMEREF{stuck}}$ does not have any computation. It is used to represent the result of
  a transition that causes the computation to terminate abruptly.

\begin{align*}
  \KEY{Entity} \quad
  & \_ \xrightarrow{\NAMEDECL{abrupted}(\_ : \NAMEHYPER{../../../Values}{Value-Types}{values}\QUERY)} \_
\end{align*}
$\SHADE{\NAMEREF{abrupted}
           (  \VAR{V} )}$ in a label on a tranistion indicates abrupt termination for
  reason $\SHADE{\VAR{V}}$. $\SHADE{\NAMEREF{abrupted}
           (   \  )}$ indicates the absence of abrupt termination.

\begin{align*}
  \KEY{Funcon} \quad
  & \NAMEDECL{finalise-abrupting}(
                       \VAR{X} :  \TO \VAR{T}) 
    :  \TO \VAR{T}  \mid \NAMEHYPER{../../../Values/Primitive}{Null}{null-type} \\&\quad
    \leadsto \NAMEREF{handle-abrupt}
               (  \VAR{X}, 
                      \NAMEHYPER{../../../Values/Primitive}{Null}{null-value} )
\end{align*}
$\SHADE{\NAMEREF{finalise-abrupting}
           (  \VAR{X} )}$ handles abrupt termination of $\SHADE{\VAR{X}}$ for any reason.

\begin{align*}
  \KEY{Funcon} \quad
  & \NAMEDECL{abrupt}(
                       \_ : \NAMEHYPER{../../../Values}{Value-Types}{values}) 
    :  \TO \NAMEHYPER{../../../Values}{Value-Types}{empty-type} 
\end{align*}
$\SHADE{\NAMEREF{abrupt}
           (  \VAR{V} )}$ terminates abruptly for reason $\SHADE{\VAR{V}}$.

\begin{align*}
  \KEY{Rule} \quad
    &  \NAMEREF{abrupt}
                    (  \VAR{V} : \NAMEHYPER{../../../Values}{Value-Types}{values} ) \xrightarrow{\NAMEREF{abrupted}(  \VAR{V} )}_{} 
        \NAMEREF{stuck}
\end{align*}
\begin{align*}
  \KEY{Funcon} \quad
  & \NAMEDECL{handle-abrupt}(
                       \_ : \VAR{T}' \TO \VAR{T}, \_ : \VAR{T}'' \TO \VAR{T}) 
    : \VAR{T}' \TO \VAR{T} 
\end{align*}
$\SHADE{\NAMEREF{handle-abrupt}
           (  \VAR{X}, 
                  \VAR{Y} )}$ first evaluates $\SHADE{\VAR{X}}$. If $\SHADE{\VAR{X}}$ terminates normally with
  value $\SHADE{\VAR{V}}$, then $\SHADE{\VAR{V}}$ is returned and $\SHADE{\VAR{Y}}$ is ignored. If $\SHADE{\VAR{X}}$ terminates abruptly
  for reason $\SHADE{\VAR{V}}$, then $\SHADE{\VAR{Y}}$ is executed with $\SHADE{\VAR{V}}$ as $\SHADE{\NAMEHYPER{../../Normal}{Giving}{given}}$ value.

$\SHADE{\NAMEREF{handle-abrupt}
           (  \VAR{X}, 
                  \VAR{Y} )}$ is associative, with $\SHADE{\NAMEREF{abrupt}
           (  \NAMEHYPER{../../Normal}{Giving}{given} )}$ as left and right
  unit. $\SHADE{\NAMEREF{handle-abrupt}
           (  \VAR{X}, 
                  \NAMEHYPER{../.}{Failing}{else}
                   (  \VAR{Y}, 
                          \NAMEREF{abrupt}
                           (  \NAMEHYPER{../../Normal}{Giving}{given} ) ) )}$ ensures propagation of 
  abrupt termination for the given reason if $\SHADE{\VAR{Y}}$ fails

\begin{align*}
  \KEY{Rule} \quad
    & \RULE{
      &  \VAR{X} \xrightarrow{\NAMEREF{abrupted}(   \  )}_{} 
          \VAR{X}'
      }{
      &  \NAMEREF{handle-abrupt}
                      (  \VAR{X}, 
                             \VAR{Y} ) \xrightarrow{\NAMEREF{abrupted}(   \  )}_{} 
          \NAMEREF{handle-abrupt}
            (  \VAR{X}', 
                   \VAR{Y} )
      }
\\
  \KEY{Rule} \quad
    & \RULE{
      &  \VAR{X} \xrightarrow{\NAMEREF{abrupted}(  \VAR{V} : \VAR{T}'' )}_{} 
          \VAR{X}'
      }{
      &  \NAMEREF{handle-abrupt}
                      (  \VAR{X}, 
                             \VAR{Y} ) \xrightarrow{\NAMEREF{abrupted}(   \  )}_{} 
          \NAMEHYPER{../../Normal}{Giving}{give}
            (  \VAR{V}, 
                   \VAR{Y} )
      }
\\
  \KEY{Rule} \quad
    & \NAMEREF{handle-abrupt}
        (  \VAR{V} : \VAR{T}, 
               \VAR{Y} ) \leadsto 
        \VAR{V}
\end{align*}
\begin{align*}
  \KEY{Funcon} \quad
  & \NAMEDECL{finally}(
                       \_ :  \TO \VAR{T}, \_ :  \TO \NAMEHYPER{../../../Values/Primitive}{Null}{null-type}) 
    :  \TO \VAR{T} 
\end{align*}
$\SHADE{\NAMEREF{finally}
           (  \VAR{X}, 
                  \VAR{Y} )}$ first executes $\SHADE{\VAR{X}}$. If $\SHADE{\VAR{X}}$ terminates normally with 
  value $\SHADE{\VAR{V}}$, then $\SHADE{\VAR{Y}}$ is executed before terminating normally with value $\SHADE{\VAR{V}}$.
  If $\SHADE{\VAR{X}}$ terminates abruptly for reason $\SHADE{\VAR{V}}$, then $\SHADE{\VAR{Y}}$ is executed before
  terminating abruptly with the same reason $\SHADE{\VAR{V}}$.

\begin{align*}
  \KEY{Rule} \quad
    & \RULE{
      &  \VAR{X} \xrightarrow{\NAMEREF{abrupted}(   \  )}_{} 
          \VAR{X}'
      }{
      &  \NAMEREF{finally}
                      (  \VAR{X}, 
                             \VAR{Y} ) \xrightarrow{\NAMEREF{abrupted}(   \  )}_{} 
          \NAMEREF{finally}
            (  \VAR{X}', 
                   \VAR{Y} )
      }
\\
  \KEY{Rule} \quad
    & \RULE{
      &  \VAR{X} \xrightarrow{\NAMEREF{abrupted}(  \VAR{V} : \NAMEHYPER{../../../Values}{Value-Types}{values} )}_{} 
          \VAR{X}'
      }{
      &  \NAMEREF{finally}
                      (  \VAR{X}, 
                             \VAR{Y} ) \xrightarrow{\NAMEREF{abrupted}(   \  )}_{} 
          \NAMEHYPER{../../Normal}{Flowing}{sequential}
            (  \VAR{Y}, 
                   \NAMEREF{abrupt}
                    (  \VAR{V} ) )
      }
\\
  \KEY{Rule} \quad
    & \NAMEREF{finally}
        (  \VAR{V} : \VAR{T}, 
               \VAR{Y} ) \leadsto 
        \NAMEHYPER{../../Normal}{Flowing}{sequential}
          (  \VAR{Y}, 
                 \VAR{V} )
\end{align*}
% 


