% 



    OUTLINE
  \tableofcontents
\begin{center}
\rule{3in}{0.4pt}
\end{center}

\subsubsection{Flowing}\hypertarget{flowing}{}\label{flowing}

\begin{align*}
  [ \
  \KEY{Funcon} \quad & \NAMEREF{left-to-right} \\
  \KEY{Alias} \quad & \NAMEREF{l-to-r} \\
  \KEY{Funcon} \quad & \NAMEREF{right-to-left} \\
  \KEY{Alias} \quad & \NAMEREF{r-to-l} \\
  \KEY{Funcon} \quad & \NAMEREF{sequential} \\
  \KEY{Alias} \quad & \NAMEREF{seq} \\
  \KEY{Funcon} \quad & \NAMEREF{effect} \\
  \KEY{Funcon} \quad & \NAMEREF{choice} \\
  \KEY{Funcon} \quad & \NAMEREF{if-true-else} \\
  \KEY{Alias} \quad & \NAMEREF{if-else} \\
  \KEY{Funcon} \quad & \NAMEREF{while-true} \\
  \KEY{Alias} \quad & \NAMEREF{while} \\
  \KEY{Funcon} \quad & \NAMEREF{do-while-true} \\
  \KEY{Alias} \quad & \NAMEREF{do-while} \\
  \KEY{Funcon} \quad & \NAMEREF{interleave} \\
  \KEY{Datatype} \quad & \NAMEREF{yielding} \\
  \KEY{Funcon} \quad & \NAMEREF{signal} \\
  \KEY{Funcon} \quad & \NAMEREF{yielded} \\
  \KEY{Funcon} \quad & \NAMEREF{yield} \\
  \KEY{Funcon} \quad & \NAMEREF{yield-on-value} \\
  \KEY{Funcon} \quad & \NAMEREF{yield-on-abrupt} \\
  \KEY{Funcon} \quad & \NAMEREF{atomic}
  \ ]
\end{align*}
\begin{align*}
  \KEY{Meta-variables} \quad
  & \VAR{T} <: \NAMEHYPER{../../../Values}{Value-Types}{values} \qquad \\& \VAR{T}\STAR <: \NAMEHYPER{../../../Values}{Value-Types}{values}\STAR
\end{align*}
\paragraph{Sequencing}\hypertarget{sequencing}{}\label{sequencing}

\begin{align*}
  \KEY{Funcon} \quad
  & \NAMEDECL{left-to-right}(
                       \_ : (   \TO (  \VAR{T} )\STAR )\STAR) 
    :  \TO (  \VAR{T} )\STAR 
\\
  \KEY{Alias} \quad
  & \NAMEDECL{l-to-r} = \NAMEREF{left-to-right}
\end{align*}
$\SHADE{\NAMEREF{left-to-right}
           (  \cdots )}$ computes its arguments sequentially, from left to right,
  and gives the resulting sequence of values, provided all terminate normally.
  For example, $\SHADE{\NAMEHYPER{../../../Values/Primitive}{Integers}{integer-add}
           (  \VAR{X}, 
                  \VAR{Y} )}$ may interleave the computations of $\SHADE{\VAR{X}}$ and
  $\SHADE{\VAR{Y}}$, whereas $\SHADE{\NAMEHYPER{../../../Values/Primitive}{Integers}{integer-add} \ 
           \NAMEREF{left-to-right}
             (  \VAR{X}, 
                    \VAR{Y} )}$ always computes $\SHADE{\VAR{X}}$ before $\SHADE{\VAR{Y}}$.

When each argument of $\SHADE{\NAMEREF{left-to-right}
           (  \cdots )}$ computes a single value, the type
  of the result is the same as that of the argument sequence. For instance,
  when $\SHADE{\VAR{X} : \VAR{T}}$ and $\SHADE{\VAR{Y} : \VAR{T}'}$, the result of $\SHADE{\NAMEREF{left-to-right}
           (  \VAR{X}, 
                  \VAR{Y} )}$ is of type $\SHADE{(  \VAR{T}, 
                \VAR{T}' )}$.
  The only effect of wrapping an argument sequence in $\SHADE{\NAMEREF{left-to-right}
           (  \cdots )}$ is to
  ensure that when the arguments are to be evaluated, it is done in the
  specified order.

\begin{align*}
  \KEY{Rule} \quad
    & \RULE{
      &  \VAR{Y} \TRANS 
          \VAR{Y}'
      }{
      &  \NAMEREF{left-to-right}
                      (  \VAR{V}\STAR : (  \VAR{T} )\STAR, 
                             \VAR{Y}, 
                             \VAR{Z}\STAR ) \TRANS 
          \NAMEREF{left-to-right}
            (  \VAR{V}\STAR, 
                   \VAR{Y}', 
                   \VAR{Z}\STAR )
      }
\\
  \KEY{Rule} \quad
    & \NAMEREF{left-to-right}
        (  \VAR{V}\STAR : (  \VAR{T} )\STAR ) \leadsto 
        \VAR{V}\STAR
\end{align*}
\begin{align*}
  \KEY{Funcon} \quad
  & \NAMEDECL{right-to-left}(
                       \_ : (   \TO (  \VAR{T} )\STAR )\STAR) 
    :  \TO (  \VAR{T} )\STAR 
\\
  \KEY{Alias} \quad
  & \NAMEDECL{r-to-l} = \NAMEREF{right-to-left}
\end{align*}
$\SHADE{\NAMEREF{right-to-left}
           (  \cdots )}$ computes its arguments sequentially, from right to left,
  and gives the resulting sequence of values, provided all terminate normally.

Note that $\SHADE{\NAMEREF{right-to-left}
           (  \VAR{X}\STAR )}$ and $\SHADE{\NAMEHYPER{../../../Values/Composite}{Sequences}{reverse} \ 
           \NAMEREF{left-to-right} \ 
             \NAMEHYPER{../../../Values/Composite}{Sequences}{reverse}
               (  \VAR{X}\STAR )}$ are
  not equivalent: $\SHADE{\NAMEHYPER{../../../Values/Composite}{Sequences}{reverse}
           (  \VAR{X}\STAR )}$ interleaves the evaluation of $\SHADE{\VAR{X}\STAR}$.

\begin{align*}
  \KEY{Rule} \quad
    & \RULE{
      &  \VAR{Y} \TRANS 
          \VAR{Y}'
      }{
      &  \NAMEREF{right-to-left}
                      (  \VAR{X}\STAR, 
                             \VAR{Y}, 
                             \VAR{V}\STAR : (  \VAR{T} )\STAR ) \TRANS 
          \NAMEREF{right-to-left}
            (  \VAR{X}\STAR, 
                   \VAR{Y}', 
                   \VAR{V}\STAR )
      }
\\
  \KEY{Rule} \quad
    & \NAMEREF{right-to-left}
        (  \VAR{V}\STAR : (  \VAR{T} )\STAR ) \leadsto 
        \VAR{V}\STAR
\end{align*}
\begin{align*}
  \KEY{Funcon} \quad
  & \NAMEDECL{sequential}(
                       \_ : (   \TO \NAMEHYPER{../../../Values/Primitive}{Null}{null-type} )\STAR, \_ :  \TO \VAR{T}) 
    :  \TO \VAR{T} 
\\
  \KEY{Alias} \quad
  & \NAMEDECL{seq} = \NAMEREF{sequential}
\end{align*}
$\SHADE{\NAMEREF{sequential}
           (  \VAR{X}, 
                  \cdots )}$ computes its arguments in the given order. On normal
  termination, it returns the value of the last argument; the other arguments
  all compute $\SHADE{\NAMEHYPER{../../../Values/Primitive}{Null}{null-value}}$.

Binary $\SHADE{\NAMEREF{sequential}
           (  \VAR{X}, 
                  \VAR{Y} )}$ is associative, with unit $\SHADE{\NAMEHYPER{../../../Values/Primitive}{Null}{null-value}}$.

\begin{align*}
  \KEY{Rule} \quad
    & \RULE{
      &  \VAR{X} \TRANS 
          \VAR{X}'
      }{
      &  \NAMEREF{sequential}
                      (  \VAR{X}, 
                             \VAR{Y}\PLUS ) \TRANS 
          \NAMEREF{sequential}
            (  \VAR{X}', 
                   \VAR{Y}\PLUS )
      }
\\
  \KEY{Rule} \quad
    & \NAMEREF{sequential}
        (  \NAMEHYPER{../../../Values/Primitive}{Null}{null-value}, 
               \VAR{Y}\PLUS ) \leadsto 
        \NAMEREF{sequential}
          (  \VAR{Y}\PLUS )
\\
  \KEY{Rule} \quad
    & \NAMEREF{sequential}
        (  \VAR{Y} ) \leadsto 
        \VAR{Y}
\end{align*}
\begin{align*}
  \KEY{Funcon} \quad
  & \NAMEDECL{effect}(
                       \VAR{V}\STAR : \VAR{T}\STAR) 
    :  \TO \NAMEHYPER{../../../Values/Primitive}{Null}{null-type} \\&\quad
    \leadsto \NAMEHYPER{../../../Values/Primitive}{Null}{null-value}
\end{align*}
$\SHADE{\NAMEREF{effect}
           (  \cdots )}$ interleaves the computations of its arguments, then discards
  all the computed values.

\paragraph{Choosing}\hypertarget{choosing}{}\label{choosing}

\begin{align*}
  \KEY{Funcon} \quad
  & \NAMEDECL{choice}(
                       \_ : (   \TO \VAR{T} )\PLUS) 
    :  \TO \VAR{T} 
\end{align*}
$\SHADE{\NAMEREF{choice}
           (  \VAR{Y}, 
                  \cdots )}$ selects one of its arguments, then computes it.
  It is associative and commutative.

\begin{align*}
  \KEY{Rule} \quad
    & \NAMEREF{choice}
        (  \VAR{X}\STAR, 
               \VAR{Y}, 
               \VAR{Z}\STAR ) \leadsto 
        \VAR{Y}
\end{align*}
\begin{align*}
  \KEY{Funcon} \quad
  & \NAMEDECL{if-true-else}(
                       \_ : \NAMEHYPER{../../../Values/Primitive}{Booleans}{booleans}, \_ :  \TO \VAR{T}, \_ :  \TO \VAR{T}) 
    :  \TO \VAR{T} 
\\
  \KEY{Alias} \quad
  & \NAMEDECL{if-else} = \NAMEREF{if-true-else}
\end{align*}
$\SHADE{\NAMEREF{if-true-else}
           (  \VAR{B}, 
                  \VAR{X}, 
                  \VAR{Y} )}$ evaluates $\SHADE{\VAR{B}}$ to a Boolean value, then reduces
  to $\SHADE{\VAR{X}}$ or $\SHADE{\VAR{Y}}$, depending on the value of $\SHADE{\VAR{B}}$.

\begin{align*}
  \KEY{Rule} \quad
    & \NAMEREF{if-true-else}
        (  \NAMEHYPER{../../../Values/Primitive}{Booleans}{true}, 
               \VAR{X}, 
               \VAR{Y} ) \leadsto 
        \VAR{X}
\\
  \KEY{Rule} \quad
    & \NAMEREF{if-true-else}
        (  \NAMEHYPER{../../../Values/Primitive}{Booleans}{false}, 
               \VAR{X}, 
               \VAR{Y} ) \leadsto 
        \VAR{Y}
\end{align*}
\paragraph{Iterating}\hypertarget{iterating}{}\label{iterating}

\begin{align*}
  \KEY{Funcon} \quad
  & \NAMEDECL{while-true}(
                       \VAR{B} :  \TO \NAMEHYPER{../../../Values/Primitive}{Booleans}{booleans}, \VAR{X} :  \TO \NAMEHYPER{../../../Values/Primitive}{Null}{null-type}) 
    :  \TO \NAMEHYPER{../../../Values/Primitive}{Null}{null-type} \\&\quad
    \leadsto \NAMEREF{if-true-else}
               (  \VAR{B}, 
                      \NAMEREF{sequential}
                       (  \VAR{X}, 
                              \NAMEREF{while-true}
                               (  \VAR{B}, 
                                      \VAR{X} ) ), 
                      \NAMEHYPER{../../../Values/Primitive}{Null}{null-value} )
\\
  \KEY{Alias} \quad
  & \NAMEDECL{while} = \NAMEREF{while-true}
\end{align*}
$\SHADE{\NAMEREF{while-true}
           (  \VAR{B}, 
                  \VAR{X} )}$ evaluates $\SHADE{\VAR{B}}$ to a Boolean value. Depending on the value
  of $\SHADE{\VAR{B}}$, it either executes $\SHADE{\VAR{X}}$ and iterates, or terminates normally.

The effect of abruptly breaking the iteration is obtained by the combination
  $\SHADE{\NAMEHYPER{../../Abnormal}{Breaking}{handle-break}
           (  \NAMEREF{while-true}
                   (  \VAR{B}, 
                          \VAR{X} ) )}$, and that of abruptly continuing the iteration by
  $\SHADE{\NAMEREF{while-true}
           (  \VAR{B}, 
                  \NAMEHYPER{../../Abnormal}{Continuing}{handle-continue}
                   (  \VAR{X} ) )}$.

\begin{align*}
  \KEY{Funcon} \quad
  & \NAMEDECL{do-while-true}(
                       \VAR{X} :  \TO \NAMEHYPER{../../../Values/Primitive}{Null}{null-type}, \VAR{B} :  \TO \NAMEHYPER{../../../Values/Primitive}{Booleans}{booleans}) 
    :  \TO \NAMEHYPER{../../../Values/Primitive}{Null}{null-type} \\&\quad
    \leadsto \NAMEREF{sequential}
               (  \VAR{X}, 
                      \NAMEREF{if-true-else}
                       (  \VAR{B}, 
                              \NAMEREF{do-while-true}
                               (  \VAR{X}, 
                                      \VAR{B} ), 
                              \NAMEHYPER{../../../Values/Primitive}{Null}{null-value} ) )
\\
  \KEY{Alias} \quad
  & \NAMEDECL{do-while} = \NAMEREF{do-while-true}
\end{align*}
$\SHADE{\NAMEREF{do-while-true}
           (  \VAR{X}, 
                  \VAR{B} )}$ is equivalent to $\SHADE{\NAMEREF{sequential}
           (  \VAR{X}, 
                  \NAMEREF{while-true}
                   (  \VAR{B}, 
                          \VAR{X} ) )}$.

\paragraph{Interleaving}\hypertarget{interleaving}{}\label{interleaving}

\begin{align*}
  \KEY{Funcon} \quad
  & \NAMEDECL{interleave}(
                       \_ : \VAR{T}\STAR) 
    :  \TO \VAR{T}\STAR 
\end{align*}
$\SHADE{\NAMEREF{interleave}
           (  \cdots )}$ computes its arguments in any order, possibly interleaved,
  and returns the resulting sequence of values, provided all terminate normally.
  Fairness of interleaving is not required, so pure left-to-right computation
  is allowed.

$\SHADE{\NAMEREF{atomic}
           (  \VAR{X} )}$ prevents interleaving in $\SHADE{\VAR{X}}$, except after transitions that emit
  a $\SHADE{\NAMEREF{yielded}
           (  \NAMEREF{signal} )}$.

\begin{align*}
  \KEY{Rule} \quad
    & \NAMEREF{interleave}
        (  \VAR{V}\STAR : \VAR{T}\STAR ) \leadsto 
        \VAR{V}\STAR
\end{align*}
\begin{align*}
  \KEY{Datatype} \quad 
  \NAMEDECL{yielding} 
  \ ::= \ & \NAMEDECL{signal}
\end{align*}
\begin{align*}
  \KEY{Entity} \quad
  & \_ \xrightarrow{\NAMEDECL{yielded}(\_ : \NAMEREF{yielding}\QUERY)} \_
\end{align*}
$\SHADE{\NAMEREF{yielded}
           (  \NAMEREF{signal} )}$ in a label on a transition allows interleaving at that point
  in the enclosing atomic computation.
  $\SHADE{\NAMEREF{yielded}
           (   \  )}$ indicates interleaving at that point in an atomic computation
  is not allowed.

\begin{align*}
  \KEY{Funcon} \quad
  & \NAMEDECL{yield} 
    :  \TO \NAMEHYPER{../../../Values/Primitive}{Null}{null-type} \\&\quad
    \leadsto \NAMEREF{yield-on-value}
               (  \NAMEHYPER{../../../Values/Primitive}{Null}{null-value} )
\end{align*}
\begin{align*}
  \KEY{Funcon} \quad
  & \NAMEDECL{yield-on-value}(
                       \_ : \VAR{T}) 
    :  \TO \VAR{T} 
\end{align*}
$\SHADE{\NAMEREF{yield-on-value}
           (  \VAR{X} )}$ allows interleaving in an enclosing atomic computation
  on normal termination of $\SHADE{\VAR{X}}$.

\begin{align*}
  \KEY{Rule} \quad
    &  \NAMEREF{yield-on-value}
                    (  \VAR{V} : \VAR{T} ) \xrightarrow{\NAMEREF{yielded}(  \NAMEREF{signal} )}_{} 
        \VAR{V}
\end{align*}
\begin{align*}
  \KEY{Funcon} \quad
  & \NAMEDECL{yield-on-abrupt}(
                       \_ :  \TO \VAR{T}) 
    :  \TO \VAR{T} 
\end{align*}
$\SHADE{\NAMEREF{yield-on-abrupt}
           (  \VAR{X} )}$ ensures that abrupt termination of $\SHADE{\VAR{X}}$ is propagated
  through an enclosing atomic computation.

\begin{align*}
  \KEY{Rule} \quad
    & \RULE{
      &  \VAR{X} \xrightarrow{\NAMEHYPER{../../Abnormal}{Abrupting}{abrupt}(  \VAR{V} : \VAR{T} ), \NAMEREF{yielded}(  \_\QUERY )}_{} 
          \VAR{X}'
      }{
      &  \NAMEREF{yield-on-abrupt}
                      (  \VAR{X} ) \xrightarrow{\NAMEHYPER{../../Abnormal}{Abrupting}{abrupt}(  \VAR{V} ), \NAMEREF{yielded}(  \NAMEREF{signal} )}_{} 
          \NAMEREF{yield-on-abrupt}
            (  \VAR{X}' )
      }
\\
  \KEY{Rule} \quad
    & \RULE{
      &  \VAR{X} \xrightarrow{\NAMEHYPER{../../Abnormal}{Abrupting}{abrupt}(   \  )}_{} 
          \VAR{X}'
      }{
      &  \NAMEREF{yield-on-abrupt}
                      (  \VAR{X} ) \xrightarrow{\NAMEHYPER{../../Abnormal}{Abrupting}{abrupt}(   \  )}_{} 
          \NAMEREF{yield-on-abrupt}
            (  \VAR{X}' )
      }
\\
  \KEY{Rule} \quad
    & \NAMEREF{yield-on-abrupt}
        (  \VAR{V} : \VAR{T} ) \leadsto 
        \VAR{V}
\end{align*}
\begin{align*}
  \KEY{Funcon} \quad
  & \NAMEDECL{atomic}(
                       \_ :  \TO \VAR{T}) 
    :  \TO \VAR{T} 
\end{align*}
$\SHADE{\NAMEREF{atomic}
           (  \VAR{X} )}$ computes $\SHADE{\VAR{X}}$, but controls its potential interleaving with other
  computations: interleaving is only allowed following a transition of $\SHADE{\VAR{X}}$ that
  emits $\SHADE{\NAMEREF{yielded}
           (  \NAMEREF{signal} )}$.

\begin{align*}
  \KEY{Rule} \quad
    & \RULE{
      &  \VAR{X} \xrightarrow{\NAMEREF{yielded}(   \  )}_{1} 
          \VAR{X}'\\&
         \NAMEREF{atomic}
                      (  \VAR{X}' ) \xrightarrow{\NAMEREF{yielded}(   \  )}_{2} 
          \VAR{X}''
      }{
      &  \NAMEREF{atomic}
                      (  \VAR{X} ) \xrightarrow{\NAMEREF{yielded}(   \  )}_{1} ; \xrightarrow{\NAMEREF{yielded}(   \  )}_{2} 
          \VAR{X}''
      }
\\
  \KEY{Rule} \quad
    & \RULE{
      &  \VAR{X} \xrightarrow{\NAMEREF{yielded}(   \  )}_{} 
          \VAR{V}\\&
         \VAR{V} : \VAR{T}
      }{
      &  \NAMEREF{atomic}
                      (  \VAR{X} ) \xrightarrow{\NAMEREF{yielded}(   \  )}_{} 
          \VAR{V}
      }
\\
  \KEY{Rule} \quad
    & \NAMEREF{atomic}
        (  \VAR{V} : \VAR{T} ) \leadsto 
        \VAR{V}
\\
  \KEY{Rule} \quad
    & \RULE{
      &  \VAR{X} \xrightarrow{\NAMEREF{yielded}(  \NAMEREF{signal} )}_{} 
          \VAR{X}'
      }{
      &  \NAMEREF{atomic}
                      (  \VAR{X} ) \xrightarrow{\NAMEREF{yielded}(   \  )}_{} 
          \NAMEREF{atomic}
            (  \VAR{X}' )
      }
\end{align*}
% 


