% 


\begin{center}
\rule{3in}{0.4pt}
\end{center}

\subsubsection{Thunks}\hypertarget{thunks}{}\label{thunks}

\begin{align*}
  [ \
  \KEY{Datatype} \quad & \NAMEREF{thunks} \\
  \KEY{Funcon} \quad & \NAMEREF{thunk} \\
  \KEY{Funcon} \quad & \NAMEREF{force}
  \ ]
\end{align*}
\begin{align*}
  \KEY{Meta-variables} \quad
  & \VAR{T} <: \NAMEHYPER{../..}{Value-Types}{values}
\end{align*}
\begin{align*}
  \KEY{Datatype} \quad 
  \NAMEDECL{thunks}(
                     \VAR{T} ) 
  \ ::= \ & \NAMEDECL{thunk}(
                               \_ : \NAMEHYPER{../.}{Generic}{abstractions}
                                         (  (   \  ) \TO \VAR{T} ))
\end{align*}
$\SHADE{\NAMEREF{thunks}
           (  \VAR{T} )}$ consists of abstractions whose bodies do not depend on
  a given value, and whose executions normally compute values of type $\SHADE{\VAR{T}}$.
  $\SHADE{\NAMEREF{thunk}
           (  \NAMEHYPER{../.}{Generic}{abstraction}
                   (  \VAR{X} ) )}$ evaluates to a thunk with dynamic bindings,
  $\SHADE{\NAMEREF{thunk}
           (  \NAMEHYPER{../.}{Generic}{closure}
                   (  \VAR{X} ) )}$ computes a thunk with static bindings.

\begin{align*}
  \KEY{Funcon} \quad
  & \NAMEDECL{force}(
                       \_ : \NAMEREF{thunks}
                                 (  \VAR{T} )) 
    :  \TO \VAR{T} 
\end{align*}
$\SHADE{\NAMEREF{force}
           (  \VAR{H} )}$ enacts the abstraction of the thunk $\SHADE{\VAR{H}}$.

\begin{align*}
  \KEY{Rule} \quad
    & \NAMEREF{force}
        (  \NAMEREF{thunk}
                (  \NAMEHYPER{../.}{Generic}{abstraction}
                        (  \VAR{X} ) ) ) \leadsto 
        \NAMEHYPER{../../../Computations/Normal}{Giving}{no-given}
          (  \VAR{X} )
\end{align*}
% 


