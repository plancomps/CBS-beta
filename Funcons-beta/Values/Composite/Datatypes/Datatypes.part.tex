% 


\begin{center}
\rule{3in}{0.4pt}
\end{center}

\subsubsection{Datatypes}\hypertarget{datatypes}{}\label{datatypes}

\begin{align*}
  [ \
  \KEY{Type} \quad & \NAMEREF{datatype-values} \\
  \KEY{Funcon} \quad & \NAMEREF{datatype-value} \\
  \KEY{Funcon} \quad & \NAMEREF{datatype-value-id} \\
  \KEY{Funcon} \quad & \NAMEREF{datatype-value-elements}
  \ ]
\end{align*}
A datatype value consists of an identifier and a sequence of values.

`Datatype T ::= \ldots{}' declares the type $\SHADE{\VAR{T}}$ to refer to a fresh value
  constructor in $\SHADE{\NAMEHYPER{../..}{Value-Types}{types}}$, and asserts $\SHADE{\VAR{T} <: \NAMEREF{datatype-values}}$.

Each constructor funcon `F(\_:T\_1,\ldots{},\_:T\_n)' of the datatype declaration
  generates values in $\SHADE{\VAR{T}}$ of the form $\SHADE{\NAMEREF{datatype-value}
           (  \STRING{F}, 
                  \VAR{V}\SUB{1}, 
                  \cdots, 
                  \VAR{V}\SUB{n} )}$ from
  $\SHADE{\VAR{V}\SUB{1} : \VAR{T}\SUB{1}}$, \ldots{}, $\SHADE{\VAR{V}\SUB{n} : \VAR{T}\SUB{n}}$.

Note that a computation $\SHADE{\VAR{X}}$ cannot be directly included in datatype values:
  it is necessary to encapsulate it in $\SHADE{\NAMEHYPER{../../Abstraction}{Generic}{abstraction}
           (  \VAR{X} )}$.

`Datatype T', followed by declarations of constructor funcons for `T',
  allows specification of GADTs.

\begin{align*}
  \KEY{Built-in Type} \quad 
  & \NAMEDECL{datatype-values}  
\end{align*}
\begin{align*}
  \KEY{Built-in Funcon} \quad
  & \NAMEDECL{datatype-value}(
                       \_ : \NAMEHYPER{../../../Computations/Normal}{Binding}{identifiers}, \_ : \NAMEHYPER{../..}{Value-Types}{values}\STAR) 
    : \NAMEREF{datatype-values} 
\end{align*}
\begin{align*}
  \KEY{Funcon} \quad
  & \NAMEDECL{datatype-value-id}(
                       \_ : \NAMEREF{datatype-values}) 
    :  \TO \NAMEHYPER{../../../Computations/Normal}{Binding}{identifiers} 
\\
  \KEY{Rule} \quad
    & \NAMEREF{datatype-value-id}
        (  \NAMEREF{datatype-value}
                (  \VAR{I} : \NAMEHYPER{../../../Computations/Normal}{Binding}{identifiers}, 
                       \_\STAR : \NAMEHYPER{../..}{Value-Types}{values}\STAR ) ) \leadsto 
        \VAR{I}
\end{align*}
\begin{align*}
  \KEY{Funcon} \quad
  & \NAMEDECL{datatype-value-elements}(
                       \_ : \NAMEREF{datatype-values}) 
    :  \TO \NAMEHYPER{../..}{Value-Types}{values}\STAR 
\\
  \KEY{Rule} \quad
    & \NAMEREF{datatype-value-elements}
        (  \NAMEREF{datatype-value}
                (  \_ : \NAMEHYPER{../../../Computations/Normal}{Binding}{identifiers}, 
                       \VAR{V}\STAR : \NAMEHYPER{../..}{Value-Types}{values}\STAR ) ) \leadsto 
        \VAR{V}\STAR
\end{align*}
% 


