% 



    OUTLINE
  \tableofcontents
\begin{center}
\rule{3in}{0.4pt}
\end{center}

\subsubsection{Interacting}\hypertarget{interacting}{}\label{interacting}

\paragraph{Output}\hypertarget{output}{}\label{output}

\begin{align*}
  [ \
  \KEY{Entity} \quad & \NAMEREF{standard-out} \\
  \KEY{Funcon} \quad & \NAMEREF{print}
  \ ]
\end{align*}
\begin{align*}
  \KEY{Entity} \quad
  & \_ \xrightarrow{\NAMEDECL{standard-out}!(\_ : \NAMEHYPER{../../../Values}{Value-Types}{values}\STAR)} \_
\end{align*}
This entity represents the sequence of values output by a particular
  transition, where the empty sequence $\SHADE{(   \  )}$ represents the lack of output.
  Composition of transitions concatenates their output sequences.

\begin{align*}
  \KEY{Funcon} \quad
  & \NAMEDECL{print}(
                       \_ : \NAMEHYPER{../../../Values}{Value-Types}{values}\STAR) 
    :  \TO \NAMEHYPER{../../../Values/Primitive}{Null}{null-type} 
\end{align*}
$\SHADE{\NAMEREF{print}
           (  \VAR{X}\STAR )}$ evaluates the arguments $\SHADE{\VAR{X}\STAR}$ and emits the resulting sequence of
  values on the standard-out channel. $\SHADE{\NAMEREF{print}
           (   \  )}$ has no effect.

\begin{align*}
  \KEY{Rule} \quad
    &  \NAMEREF{print}
                    (  \VAR{V}\STAR : \NAMEHYPER{../../../Values}{Value-Types}{values}\STAR ) \xrightarrow{\NAMEREF{standard-out}!(  \VAR{V}\STAR )}_{} 
        \NAMEHYPER{../../../Values/Primitive}{Null}{null-value}
\end{align*}
\paragraph{Input}\hypertarget{input}{}\label{input}

\begin{align*}
  [ \
  \KEY{Entity} \quad & \NAMEREF{standard-in} \\
  \KEY{Funcon} \quad & \NAMEREF{read}
  \ ]
\end{align*}
\begin{align*}
  \KEY{Entity} \quad
  & \_ \xrightarrow{\NAMEDECL{standard-in}?(\_ : \NAMEHYPER{../../../Values}{Value-Types}{values}\STAR)} \_
\end{align*}
This entity represents the sequence of values input by a particular
  transition, where the empty sequence $\SHADE{(   \  )}$ represents that no values are
  input. The value $\SHADE{\NAMEHYPER{../../../Values/Primitive}{Null}{null-value}}$ represents the end of the input.

Composition of transitions concatenates their input sequences, except that
  when the first sequence ends with $\SHADE{\NAMEHYPER{../../../Values/Primitive}{Null}{null-value}}$, the second seqeunce has to be
  just $\SHADE{\NAMEHYPER{../../../Values/Primitive}{Null}{null-value}}$.

\begin{align*}
  \KEY{Funcon} \quad
  & \NAMEDECL{read} 
    :  \TO \NAMEHYPER{../../../Values}{Value-Types}{values} 
\end{align*}
$\SHADE{\NAMEREF{read}}$ inputs a single value from the standard-in channel, and returns it.
  If the end of the input has been reached, $\SHADE{\NAMEREF{read}}$ fails.

\begin{align*}
  \KEY{Rule} \quad
    &  \NAMEREF{read} \xrightarrow{\NAMEREF{standard-in}?(  \VAR{V} : \mathop{\sim} \NAMEHYPER{../../../Values/Primitive}{Null}{null-type} )}_{} 
        \VAR{V}
\\
  \KEY{Rule} \quad
    &  \NAMEREF{read} \xrightarrow{\NAMEREF{standard-in}?(  \NAMEHYPER{../../../Values/Primitive}{Null}{null-value} )}_{} 
        \NAMEHYPER{../../Abnormal}{Failing}{fail}
\end{align*}
% 


